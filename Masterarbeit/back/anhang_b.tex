\chapter{Inhalt der CD-ROM}
\label{app:cdrom}

\paragraph{Format:} 
		CD-ROM, Single Layer, ISO9660-Format


\section{PDF-Dateien}
\begin{FileList}{/1_PDF-Dateien}
\fitem{Bachelorarbeit.pdf} Bachelorarbeit inkl. Anhang
\end{FileList}

\section{Excel-Dateien}

\begin{FileList}{/2_Excel-Dateien}
\fitem{1_AlphaVergleich.xlsx} Vergleich Brandintensitätskoeffizienten
\fitem{2_AltBrandherd.xlsx} Vergleich max. spez. Wärmefreisetzungsrate
\fitem{3_AlphaBer.xlsx} Berechnung Brandintensitätskoeffizienten
\fitem{4_C-Faktor.xlsx} Untersuchung C-Faktor
\fitem{5_MSS_H=3m.xlsx} MSS für 3 m Raum
\fitem{6_MSS_H=6m.xlsx} MSS für 6 m Raum
\fitem{7_HRRKurve.xlsx} Untersuchung $\alpha \cdot t^2$-Kurve
\fitem{8_HöhenVergleich.xlsx} Vergleich Raumhöhen
\fitem{9_RTIVergleich.xlsx} Vergleich RTI
\end{FileList}







\newpage
\section{FDS-Dateien}

\textbf{Hinweis:} Alle FDS-Inputdateien haben die Bezeichnung "`OffenerRaum.fds"'. Im folgenden wird nur die Ordnerstruktur aufgelistet.
\SuperPar
\dirtree{%
.1 /3\_FDS-Inputdateien.
.2 1\_Brandintensitõtskoeffizient.
.3 0,012.
.3 0,188.
.2 2\_Alt.\_Brandherd.
.3 1\_3m.
.3 2\_6m.
.3 3\_8m.
.2 3\_C-Faktor.
.3 1\_3m.
.4 1\_alpha=0,012.
.4 2\_alpha=0,047.
.4 3\_alpha=0,188.
.3 2\_6m.
.3 3\_8m.
.3 4\_RTI.
.4 1\_RTI=27.
.4 2\_RTI=120.
.2 4\_Plot3D.
.2 5\_Raumhöhen.
.3 1\_6m.
.3 2\_8m.
.2 6\_RTI.
.3 1\_RTI=27.
.3 2\_RTI=50.
.3 3\_RTI=80.
.3 4\_RTI=120.
.3 5\_RTI=180.
.2 7\_MSS.
.3 1\_3m.
.4 1\_2,5cm.
.4 2\_5cm.
.4 3\_10cm.
.4 4\_20cm.
.3 2\_6m.
.4 1\_5cm.
.4 2\_10cm.
.4 3\_20cm.
.2 8\_VergleichGridUeberFeuer.
.3 1\_3m.
.3 2\_6m.
}





