\chapter{Kurzfassung}

In der vorliegenden Arbeit werden Sprinklerszenarien mit der CFD-Software FDS (Fire Dynamics Simulator) simuliert. Das Simulationsmodell ist ein Raum mit unendlicher Decke und offenen Seiten. Es wird betrachtet, wie verschiedene Parameter (Raumhöhe, Brandintensitätskoeffizient, RTI) Einfluss auf die Sprinklerauslösezeit nehmen. Der Brand wird mit quadratischer Zunahme der Wärmefreisetzungrate und Brandfläche simuliert. 
Bestehende Literatur wird betrachtet und verschiedene Rechenansätze werden angeführt. Daten aus FDS werden visuell ausgewertet und interpretiert. 
Die Ergebnisse werden mit den Tabellenwerten der VDI 6019:2006 Blatt 1 und dem SFPE Handbook of Fire Protection Engineering, 3rd Edition verglichen. Es stellt sich heraus, dass die VDI~6019 nicht mehr dem derzeitigen Wissenschaftsstand widerspiegelt. Nach einer Untersuchung der Kennwerte in FDS und dem Vergleich zu Literaturquellen wird ermittelt, dass ein C-Faktor von 0~(m/s)$^{0,5}$ die Sprinklerauslösezeiten am besten vorhersagt. Übereinstimmungen der Auslösezeiten zwischen Simulation und Berechnung sind sehr hoch, vor allem bei niedrigen Nennöffnungstemperaturen.  










