\chapter{Einleitung}
\label{cha:Einleitung}

Der Brandschutz nimmt in der Gebäudetechnik einen essentiellen Teil ein. Beim vorbeugenden Brandschutz wird durch baulichen, anlagentechnischen und organisatorischen Brandschutz versucht, die potentielle Gefährdung für Mensch und Leben in Gebäuden bei einem Brand zu minimieren. Das Risiko für den Menschen geht hierbei meist durch die gefährliche Rauchentwicklung aus. 
So liegt die Quote der Todesfälle durch Rauchvergiftung und Erstickung bei Bränden in Gebäuden bei 90 \%. Die Wichtigkeit, mithilfe des anlagentechnischen Brandschutzes das Risiko für den Menschen zu minimieren, steht somit außer Frage. Entrauchungsanlagen sorgen dafür, dass der beim Brand entstandene Rauch aus dem Gebäude geleitet wird und Flucht- und Rettungswege gesichert werden. 
Zusätzlich wird oftmals vorgeschrieben, automatische Sprinklersysteme vorzusehen. Löschwasser begrenzt nach der Aktivierung durch gleichmäßiges Verteilen die Brandausbreitung und Rauchfreisetzung. Eine weitere grundsätzliche Aufgabe ist die nachgeschaltete Alarmierung der Feuerwehr nach der Auslösung. Hier ist es also wichtig, dass das Sprinklersystem schnell und zuverlässig auslöst. 

Heutzutage erforderten die Größe, Komplexität und Einzigartigkeit neuartiger Gebäude die Anwendung innovativer Methoden, um den brandschutztechnischen Nachweis, trotz Abweichungen von den bestehenden baurechtlichen Vorgaben, erbringen zu können. 
Dies erfolgt des Öfteren mithilfe von Computermodellen. Sogenannte CFD (Computational Fluid Dynamics)-Software benutzt numerische Methoden, um komplexe Strömungsmechanik simulieren zu können (wie \zB Entrauchungssimulationen). 

Um Entrauchungsanlagen richtig zu dimensionieren, ist es essentiell, die Sprinklerauslösezeiten so genau wie möglich vorhersagen zu können. Da die Sprinkleranlage das Feuer an der Ausbreitung verhindert, kann ausgesagt werden, wann der Brandherd eine bestimmte Leistung erreicht und damit auch wieviel Rauch bis zu diesem Zeitpunkt freigesetzt wurde und wieviel Rauch nachfolgend noch freigesetzt wird. Längere Sprinklerauslösezeiten führen also zu einer größeren Brandausbreitung als kürzere Auslösezeiten. Dies kann schlussendlich erhebliche Auswirkungen auf die Größe der Entrauchungsanlage haben. Die VDI-Richtlinie 6019:2006 "`Ingenieurmethoden zur Auslegung von Entrauchungsanlagen"' beschreibt diesen Sachverhalt.

In dieser Arbeit wird untersucht, wie präzise das Simulationsprogramm FDS (Fire Dynamics Simulator) Sprinklerauslösezeiten vorhersagen kann. Dies beinhaltet den Vergleich der Ergebnisse mit der VDI 6019:2006 und dem SFPE Handbook 3rd Ed. In FDS wird ein realistischer Brandherd in einem Raum mit offener Decke simuliert. Verschiedene Kennwerte der Sprinklerköpfe werden außerdem untersucht. 

Diese Arbeit wurde im Unternehmen ROM Technik in der Abteilung Forschung und Entwicklung im wärme- und strömungstechnischen Labor in Hamburg angefertigt.

%%\SuperPar In dieser Arbeit wird nach der Aufgabenstellung zunächst im Grundlagenkapitel auf die verschiedenen Einflussfaktoren im Bezug auf Sprinklerauslösezeiten eingegangen. Anschließend beschreibt der Hauptteil alle notwendigen Rechnungen und die Simulationssoftware FDS. Es wird auf das Simulationsmodell eingegangen, sowie alle Vorüberlegungen, die zum endgültigen Modell geführt haben. Darauf folgend werden alle Versuchsergebnisse vorgestellt und ausgewertet. Im Kapitel "`Fazit"' werden die Ergebnisse dieser Arbeit in die vorhandene Literatur eingeordnet und abschließende Gedanken zu finden sein. Im Anhang sind alle für die Simulationen benutzte Input-Dateien enthalten.
