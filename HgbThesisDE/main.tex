%%% Dateikodierung: UTF-8

%%% Magic Comments zum Setzen der korrekten Parameter in kompatiblen IDEs
% !TeX encoding = utf8
% !TeX program = pdflatex 
% !TeX spellcheck = de_DE
% !BIB program = biber

\RequirePackage[utf8]{inputenc} % bei Verw. von lualatex oder xelatex entfernen!
% \RequirePackage{hgbpdfa}        % Erzeugt ein PDF/A-2b-konformes Dokument

\documentclass[type=master,theme=default,language=german,titlelanguage=german,smartquotes]{hgbthesis}
% Zulässige Optionen in [..]: 
%    Typ der Arbeit (type=): 'master' (default), 'bachelor', 'diploma', 'phd', 'internship'
%    Theme der Titelseite (theme=): 'default' (default), 'fhooe24'
%    Als Exposé verwenden: 'proposal' oder 'proposal=true' 
%    Hauptsprache im Dokument (language=): 'german' (default), 'english'
%    Sprache der Titelseite (titlelanguage=): 'german', 'english' (default is main language)
%    Umwandlung in typografische Anführungszeichen: 'smartquotes'
%    APA Zitierstil: 'apa'
%    Layout: 'oneside' (einseitig, default), 'twoside' (zweiseitig)
%%%-----------------------------------------------------------------------------

\graphicspath{{images/}}  % Verzeichnis mit Bildern und Grafiken
\bibliography{references} % Biblatex-Literaturdatei (references.bib)

%%%-----------------------------------------------------------------------------
\begin{document}
%%%-----------------------------------------------------------------------------

%%%-----------------------------------------------------------------------------
% Angaben für die Titelei (Titelseite, Erklärung etc.)
%%%-----------------------------------------------------------------------------

\title{Partielle Lösungen zur "allgemeinen" Problematik}
\subtitle{Eine grundlegende Einführung}
\author{Alex A.\ Schlaumeier}

\programtype{Fachhochschul-Masterstudiengang} % oder Fachhochschul-Bachelorstudiengang
\programname{Universal Computing}
\institution{Fachhochschule Oberösterreich}

\placeofstudy{Hagenberg}
\dateofsubmission{2025}{07}{01} % {JJJJ}{MM}{TT}

% Liste der Betreuungspersonen, bis zu 4 sind möglich, Titel in [] ist optional
\advisor{Dr.~Alois B.~Treuer}
%\advisor[Zweitbetreuerin]{FH-Prof.\textsuperscript{in} Susanna A.~D. Visor, PhD}

\license{cc}      % Unter Creative Commons Lizenz veröffentlichen (empfohlen)
%\license{strict} % Restriktieve Lizenz, "Alle Rechte vorbehalten"

%%%-----------------------------------------------------------------------------
\frontmatter                                       % Titelei (röm. Seitenzahlen)
%%%-----------------------------------------------------------------------------

\maketitle
\tableofcontents

\chapter{Vorwort}

 % Ein Vorwort ist optional
\chapter{Kurzfassung}

In der vorliegenden Arbeit werden Sprinklerszenarien mit der CFD-Software FDS (Fire Dynamics Simulator) simuliert. Das Simulationsmodell ist ein Raum mit unendlicher Decke und offenen Seiten. Es wird betrachtet, wie verschiedene Parameter (Raumhöhe, Brandintensitätskoeffizient, RTI) Einfluss auf die Sprinklerauslösezeit nehmen. Der Brand wird mit quadratischer Zunahme der Wärmefreisetzungrate und Brandfläche simuliert. 
Bestehende Literatur wird betrachtet und verschiedene Rechenansätze werden angeführt. Daten aus FDS werden visuell ausgewertet und interpretiert. 
Die Ergebnisse werden mit den Tabellenwerten der VDI 6019:2006 Blatt 1 und dem SFPE Handbook of Fire Protection Engineering, 3rd Edition verglichen. Es stellt sich heraus, dass die VDI~6019 nicht mehr dem derzeitigen Wissenschaftsstand widerspiegelt. Nach einer Untersuchung der Kennwerte in FDS und dem Vergleich zu Literaturquellen wird ermittelt, dass ein C-Faktor von 0~(m/s)$^{0,5}$ die Sprinklerauslösezeiten am besten vorhersagt. Übereinstimmungen der Auslösezeiten zwischen Simulation und Berechnung sind sehr hoch, vor allem bei niedrigen Nennöffnungstemperaturen.  










		
\chapter{Abstract}

\begin{english} %switch to English language rules
This should be a 1-page (maximum) summary of your work in English.
%und hier geht dann das Abstract weiter...
\end{english}

Im englischen Abstract sollte inhaltlich das Gleiche
stehen wie in der deutschen Kurzfassung. Versuchen Sie daher, die
Kurzfassung prä\-zise umzusetzen, ohne aber dabei Wort für Wort zu
übersetzen. Beachten Sie bei der Übersetzung, dass gewisse
Redewendungen aus dem Deutschen im Englischen kein Pendant haben
oder völlig anders formuliert werden müssen und dass die
Satzstellung im Englischen sich (bekanntlich) vom Deutschen stark
unterscheidet (mehr dazu in Abschn.\ \ref{sec:englisch}). Es
empfiehlt sich übrigens -- auch bei höchstem Vertrauen in die
persönlichen Englischkenntnisse -- eine kundige Person für das
"`proof reading"' zu engagieren.

Die richtige Übersetzung für "`Diplomarbeit"' ist übrigens
schlicht \emph{thesis}, allenfalls  "`diploma thesis"' oder "`Master's thesis"', 
auf keinen Fall aber "`diploma work"' oder gar "`dissertation"'. 
Für "`Bachelorarbeit"' ist wohl "`Bachelor thesis"' die passende Übersetzung. 

Übrigens sollte für diesen Abschnitt die \emph{Spracheinstellung} in \latex\ von Deutsch
auf Englisch umgeschaltet werden, um die richtige Form der
Silbentrennung zu erhalten, die richtigen Anführungszeichen müssen allerdings selbst gesetzt werden %
(s.\ dazu die Abschnitte \ref{sec:sprachumschaltung} %
und \ref{sec:anfuehrungszeichen}).


%%%-----------------------------------------------------------------------------
\mainmatter                             % Hauptteil (ab hier arab. Seitenzahlen)
%%%-----------------------------------------------------------------------------

\chapter{Einleitung}
\label{cha:Einleitung}

Der Brandschutz nimmt in der Gebäudetechnik einen essentiellen Teil ein. Beim vorbeugenden Brandschutz wird durch baulichen, anlagentechnischen und organisatorischen Brandschutz versucht, die potentielle Gefährdung für Mensch und Leben in Gebäuden bei einem Brand zu minimieren. Das Risiko für den Menschen geht hierbei meist durch die gefährliche Rauchentwicklung aus. 
So liegt die Quote der Todesfälle durch Rauchvergiftung und Erstickung bei Bränden in Gebäuden bei 90 \%. Die Wichtigkeit, mithilfe des anlagentechnischen Brandschutzes das Risiko für den Menschen zu minimieren, steht somit außer Frage. Entrauchungsanlagen sorgen dafür, dass der beim Brand entstandene Rauch aus dem Gebäude geleitet wird und Flucht- und Rettungswege gesichert werden. 
Zusätzlich wird oftmals vorgeschrieben, automatische Sprinklersysteme vorzusehen. Löschwasser begrenzt nach der Aktivierung durch gleichmäßiges Verteilen die Brandausbreitung und Rauchfreisetzung. Eine weitere grundsätzliche Aufgabe ist die nachgeschaltete Alarmierung der Feuerwehr nach der Auslösung. Hier ist es also wichtig, dass das Sprinklersystem schnell und zuverlässig auslöst. 

Heutzutage erforderten die Größe, Komplexität und Einzigartigkeit neuartiger Gebäude die Anwendung innovativer Methoden, um den brandschutztechnischen Nachweis, trotz Abweichungen von den bestehenden baurechtlichen Vorgaben, erbringen zu können. 
Dies erfolgt des Öfteren mithilfe von Computermodellen. Sogenannte CFD (Computational Fluid Dynamics)-Software benutzt numerische Methoden, um komplexe Strömungsmechanik simulieren zu können (wie \zB Entrauchungssimulationen). 

Um Entrauchungsanlagen richtig zu dimensionieren, ist es essentiell, die Sprinklerauslösezeiten so genau wie möglich vorhersagen zu können. Da die Sprinkleranlage das Feuer an der Ausbreitung verhindert, kann ausgesagt werden, wann der Brandherd eine bestimmte Leistung erreicht und damit auch wieviel Rauch bis zu diesem Zeitpunkt freigesetzt wurde und wieviel Rauch nachfolgend noch freigesetzt wird. Längere Sprinklerauslösezeiten führen also zu einer größeren Brandausbreitung als kürzere Auslösezeiten. Dies kann schlussendlich erhebliche Auswirkungen auf die Größe der Entrauchungsanlage haben. Die VDI-Richtlinie 6019:2006 "`Ingenieurmethoden zur Auslegung von Entrauchungsanlagen"' beschreibt diesen Sachverhalt.

In dieser Arbeit wird untersucht, wie präzise das Simulationsprogramm FDS (Fire Dynamics Simulator) Sprinklerauslösezeiten vorhersagen kann. Dies beinhaltet den Vergleich der Ergebnisse mit der VDI 6019:2006 und dem SFPE Handbook 3rd Ed. In FDS wird ein realistischer Brandherd in einem Raum mit offener Decke simuliert. Verschiedene Kennwerte der Sprinklerköpfe werden außerdem untersucht. 

Diese Arbeit wurde im Unternehmen ROM Technik in der Abteilung Forschung und Entwicklung im wärme- und strömungstechnischen Labor in Hamburg angefertigt.

%%\SuperPar In dieser Arbeit wird nach der Aufgabenstellung zunächst im Grundlagenkapitel auf die verschiedenen Einflussfaktoren im Bezug auf Sprinklerauslösezeiten eingegangen. Anschließend beschreibt der Hauptteil alle notwendigen Rechnungen und die Simulationssoftware FDS. Es wird auf das Simulationsmodell eingegangen, sowie alle Vorüberlegungen, die zum endgültigen Modell geführt haben. Darauf folgend werden alle Versuchsergebnisse vorgestellt und ausgewertet. Im Kapitel "`Fazit"' werden die Ergebnisse dieser Arbeit in die vorhandene Literatur eingeordnet und abschließende Gedanken zu finden sein. Im Anhang sind alle für die Simulationen benutzte Input-Dateien enthalten.

\chapter{Die Abschlussarbeit}
\label{cha:Abschlussarbeit}


\chapter{Zum Arbeiten mit \latex}
\label{cha:ArbeitenMitLatex}


\chapter{Abbildungen, Tabellen, Quellcode}
\label{cha:Abbildungen}



\chapter[Mathem.\ Formeln etc.]{Mathematische Formeln, Gleichungen und Algorithmen}
\label{cha:Mathematik}


\chapter[Umgang mit Literatur]{Umgang mit Literatur und anderen Quellen}
\label{cha:Literatur}


Just a single test citation: \cite{Drake1948}.



\chapter{Drucken der Abschlussarbeit}
\label{cha:Drucken}


\chapter{Schlussbemerkungen}
\label{cha:Schluss}



%%%-----------------------------------------------------------------------------
\appendix                                                               % Anhang 
%%%-----------------------------------------------------------------------------

\chapter{FDS Input-Dateien}
\label{ch:InputDateien}

\section{Brandintensitätskoeffizient}
\subsection*{Brandintensitätskoeffizient gleich 0,012~kW/s²}
\begin{lstlisting}[emptylines=0,basicstyle=\tiny]
&HEAD CHID='OffenerRaum', TITLE='0,012 bis 400s' /

&MESH ID='mesh1', COLOR='MAROON', IJK=56,56,30, XB=1.,3.8,6.2,9.,0.,1.5, MPI_PROCESS=0 /
&MESH ID='mesh2', COLOR='MELON',IJK=48,56,30, XB=3.8,6.2,6.2,9.,0.,1.5, MPI_PROCESS=1 /
&MESH ID='mesh3', COLOR='MINT', IJK=56,56,30, XB=6.2,9.,6.2,9.,0.,1.5, MPI_PROCESS=2 /
&MESH ID='mesh4', COLOR='OLIVE', IJK=56,48,30, XB=1.,3.8,3.8,6.2,0.,1.5, MPI_PROCESS=3 /
&MESH ID='mesh5', COLOR='ORCHID', IJK=56,48,30, XB=6.2,9.,3.8,6.2,0.,1.5, MPI_PROCESS=4 /
&MESH ID='mesh6', COLOR='SALMON', IJK=56,56,30, XB=1.,3.8,1.,3.8,0.,1.5, MPI_PROCESS=5 /
&MESH ID='mesh7', COLOR='STEEL BLUE', IJK=48,56,30, XB=3.8,6.2,1.,3.8,0.,1.5, MPI_PROCESS=6 /
&MESH ID='mesh8', COLOR='FLESH', IJK=56,56,30, XB=6.2,9.,1.,3.8,0.,1.5, MPI_PROCESS=7 /
&MESH ID='mesh9', COLOR='BLUE', IJK=48,48,60, XB=3.8,6.2,3.8,6.2,0.,3., MPI_PROCESS=8 /
&MESH ID='mesh10', COLOR='CHOCOLATE', IJK=56,56,30, XB=6.2,9.,1.,3.8,1.5,3., MPI_PROCESS=9 /
&MESH ID='mesh11', COLOR='COBALT', IJK=56,56,30, XB=1.,3.8,6.2,9.,1.5,3., MPI_PROCESS=10 /
&MESH ID='mesh12', COLOR='HOT PINK',IJK=48,56,30, XB=3.8,6.2,6.2,9.,1.5,3., MPI_PROCESS=11 /
&MESH ID='mesh13', COLOR='KELLY GREEN', IJK=56,56,30, XB=6.2,9.,6.2,9.,1.5,3., MPI_PROCESS=12 /
&MESH ID='mesh14', COLOR='TEAL', IJK=56,48,30, XB=1.,3.8,3.8,6.2,1.5,3., MPI_PROCESS=13 /
&MESH ID='mesh15', COLOR='YELLOW', IJK=56,48,30, XB=6.2,9.,3.8,6.2,1.5,3., MPI_PROCESS=14 /
&MESH ID='mesh16', COLOR='BROWN', IJK=56,56,30, XB=1.,3.8,1.,3.8,1.5,3., MPI_PROCESS=15 /
&MESH ID='mesh17', COLOR='CADET BLUE', IJK=48,56,30, XB=3.8,6.2,1.,3.8,1.5,3., MPI_PROCESS=16 /

&MISC VERBOSE=.TRUE./
&RADI RADIATION=.FALSE. /

&TIME T_END=400. /


&REAC FUEL = 'METHANE', SOOT_YIELD = 0.2, CO_YIELD = 0.10, RADIATIVE_FRACTION=0.3 /

&SURF ID='BURNER', HRRPUA=1086.5, TAU_MF=0.01 /
&VENT XB=4.25,5.75,4.25,5.75,0.00,0.00, XYZ=5.0,5.0,0.00, RADIUS=0.75, SPREAD_RATE=0.001875, COLOR='RED', SURF_ID='BURNER' /


&VENT MB='XMIN', SURF_ID='OPEN' /  
&VENT MB='XMAX', SURF_ID='OPEN' /  
&VENT MB='YMIN', SURF_ID='OPEN' /  
&VENT MB='YMAX', SURF_ID='OPEN' / 
 
&SPEC ID='WATER VAPOR' /
&PART ID='my droplets', DIAMETER=1000., SPEC_ID='WATER VAPOR' /
&PROP ID='K-11', QUANTITY='SPRINKLER LINK TEMPERATURE', RTI=50., C_FACTOR=0.0, ACTIVATION_TEMPERATURE=680., PART_ID='my droplets', FLOW_RATE=300.0, PARTICLE_VELOCITY=10., SMOKEVIEW_ID='sprinkler_pendent' /
&DEVC ID='Spr-1', XYZ=5.0,1.75,2.97, PROP_ID='K-11' /

&DEVC XYZ=5.0,1.75,2.97, QUANTITY='TEMPERATURE', ID='T-1'/
&DEVC XYZ=5.0,1.75,2.97, QUANTITY='VELOCITY', ID='U-1'/


&CTRL ID='kill', FUNCTION_TYPE='KILL', INPUT_ID='delay' /
&CTRL ID='delay', FUNCTION_TYPE='TIME_DELAY', INPUT_ID='Spr-1', DELAY=5. /

&DUMP DT_HRR=1.0, DT_BNDF=1.0, DT_DEVC=1.0 /

&SLCF PBX=5., QUANTITY='TEMPERATURE' /
&SLCF PBZ=2.95, QUANTITY='TEMPERATURE' /
&SLCF PBX=5., QUANTITY='VELOCITY' /



&TAIL /


\end{lstlisting}

\subsection*{Brandintensitätskoeffizient gleich 0,188~kW/s²}
\begin{lstlisting}[emptylines=0,basicstyle=\tiny]
&HEAD CHID='OffenerRaum', TITLE='0,188 bis 300s' /

&MESH ID='mesh1', COLOR='MAROON', IJK=56,56,30, XB=1.,3.8,6.2,9.,0.,1.5, MPI_PROCESS=0 /
&MESH ID='mesh2', COLOR='MELON',IJK=48,56,30, XB=3.8,6.2,6.2,9.,0.,1.5, MPI_PROCESS=1 /
&MESH ID='mesh3', COLOR='MINT', IJK=56,56,30, XB=6.2,9.,6.2,9.,0.,1.5, MPI_PROCESS=2 /
&MESH ID='mesh4', COLOR='OLIVE', IJK=56,48,30, XB=1.,3.8,3.8,6.2,0.,1.5, MPI_PROCESS=3 /
&MESH ID='mesh5', COLOR='ORCHID', IJK=56,48,30, XB=6.2,9.,3.8,6.2,0.,1.5, MPI_PROCESS=4 /
&MESH ID='mesh6', COLOR='SALMON', IJK=56,56,30, XB=1.,3.8,1.,3.8,0.,1.5, MPI_PROCESS=5 /
&MESH ID='mesh7', COLOR='STEEL BLUE', IJK=48,56,30, XB=3.8,6.2,1.,3.8,0.,1.5, MPI_PROCESS=6 /
&MESH ID='mesh8', COLOR='FLESH', IJK=56,56,30, XB=6.2,9.,1.,3.8,0.,1.5, MPI_PROCESS=7 /
&MESH ID='mesh9', COLOR='BLUE', IJK=48,48,60, XB=3.8,6.2,3.8,6.2,0.,3., MPI_PROCESS=8 /
&MESH ID='mesh10', COLOR='CHOCOLATE', IJK=56,56,30, XB=6.2,9.,1.,3.8,1.5,3., MPI_PROCESS=9 /
&MESH ID='mesh11', COLOR='COBALT', IJK=56,56,30, XB=1.,3.8,6.2,9.,1.5,3., MPI_PROCESS=10 /
&MESH ID='mesh12', COLOR='HOT PINK',IJK=48,56,30, XB=3.8,6.2,6.2,9.,1.5,3., MPI_PROCESS=11 /
&MESH ID='mesh13', COLOR='KELLY GREEN', IJK=56,56,30, XB=6.2,9.,6.2,9.,1.5,3., MPI_PROCESS=12 /
&MESH ID='mesh14', COLOR='TEAL', IJK=56,48,30, XB=1.,3.8,3.8,6.2,1.5,3., MPI_PROCESS=13 /
&MESH ID='mesh15', COLOR='YELLOW', IJK=56,48,30, XB=6.2,9.,3.8,6.2,1.5,3., MPI_PROCESS=14 /
&MESH ID='mesh16', COLOR='BROWN', IJK=56,56,30, XB=1.,3.8,1.,3.8,1.5,3., MPI_PROCESS=15 /
&MESH ID='mesh17', COLOR='CADET BLUE', IJK=48,56,30, XB=3.8,6.2,1.,3.8,1.5,3., MPI_PROCESS=16 /
&MISC VERBOSE=.TRUE./
&RADI RADIATION=.FALSE. /

&TIME T_END=300. /


&REAC FUEL = 'METHANE', SOOT_YIELD = 0.2, CO_YIELD = 0.10, RADIATIVE_FRACTION=0.3 /

&SURF ID='BURNER', HRRPUA=21543., TAU_MF=0.01 /
&VENT XB=4.5,5.5,4.5,5.5,0.00,0.00, XYZ=5.0,5.0,0.00, RADIUS=0.5, SPREAD_RATE=0.001666, COLOR='RED', SURF_ID='BURNER' /


&VENT MB='XMIN', SURF_ID='OPEN' /  
&VENT MB='XMAX', SURF_ID='OPEN' /  
&VENT MB='YMIN', SURF_ID='OPEN' /  
&VENT MB='YMAX', SURF_ID='OPEN' / 
 
&SPEC ID='WATER VAPOR' /
&PART ID='my droplets', DIAMETER=1000., SPEC_ID='WATER VAPOR' /
&PROP ID='K-11', QUANTITY='SPRINKLER LINK TEMPERATURE', RTI=50., C_FACTOR=0.0, ACTIVATION_TEMPERATURE=6800., PART_ID='my droplets', FLOW_RATE=300.0, PARTICLE_VELOCITY=10., SMOKEVIEW_ID='sprinkler_pendent' /
&DEVC ID='Spr-1', XYZ=5.0,1.75,2.97, PROP_ID='K-11' /

&DEVC XYZ=5.0,1.75,2.97, QUANTITY='TEMPERATURE', ID='T-1'/
&DEVC XYZ=5.0,1.75,2.97, QUANTITY='VELOCITY', ID='U-1'/


&CTRL ID='kill', FUNCTION_TYPE='KILL', INPUT_ID='delay' /
&CTRL ID='delay', FUNCTION_TYPE='TIME_DELAY', INPUT_ID='Spr-1', DELAY=5. /


&DUMP DT_HRR=1.0, DT_BNDF=1.0, DT_DEVC=1.0 /

&SLCF PBX=5., QUANTITY='TEMPERATURE' /
&SLCF PBZ=2.95, QUANTITY='TEMPERATURE' /
&SLCF PBX=5., QUANTITY='VELOCITY' /



&TAIL /


\end{lstlisting}
%%%%%%%%%%%%%%%%%%%%%%%%%%%%%%%%%%%%%%%%%%%%%%%%%%%%%%%%%%%%%%%%%
\section{Brandherd mit 500 kW/m² max. spez. Wärmeleistung}
\subsection*{Raumhöhe gleich 3 m}
\begin{lstlisting}[emptylines=0,basicstyle=\tiny]
&HEAD CHID='OffenerRaum', TITLE='H=3m, a=0,047, C=0 mit alt Brandherd' /

&MESH ID='mesh1', COLOR='MAROON', IJK=56,56,30, XB=1.,3.8,6.2,9.,0.,1.5, MPI_PROCESS=0 /
&MESH ID='mesh2', COLOR='MELON',IJK=48,56,30, XB=3.8,6.2,6.2,9.,0.,1.5, MPI_PROCESS=1 /
&MESH ID='mesh3', COLOR='MINT', IJK=56,56,30, XB=6.2,9.,6.2,9.,0.,1.5, MPI_PROCESS=2 /
&MESH ID='mesh4', COLOR='OLIVE', IJK=56,48,30, XB=1.,3.8,3.8,6.2,0.,1.5, MPI_PROCESS=3 /
&MESH ID='mesh5', COLOR='ORCHID', IJK=56,48,30, XB=6.2,9.,3.8,6.2,0.,1.5, MPI_PROCESS=4 /
&MESH ID='mesh6', COLOR='SALMON', IJK=56,56,30, XB=1.,3.8,1.,3.8,0.,1.5, MPI_PROCESS=5 /
&MESH ID='mesh7', COLOR='STEEL BLUE', IJK=48,56,30, XB=3.8,6.2,1.,3.8,0.,1.5, MPI_PROCESS=6 /
&MESH ID='mesh8', COLOR='FLESH', IJK=56,56,30, XB=6.2,9.,1.,3.8,0.,1.5, MPI_PROCESS=7 /
&MESH ID='mesh9', COLOR='BLUE', IJK=48,48,60, XB=3.8,6.2,3.8,6.2,0.,3., MPI_PROCESS=8 /
&MESH ID='mesh10', COLOR='CHOCOLATE', IJK=56,56,30, XB=6.2,9.,1.,3.8,1.5,3., MPI_PROCESS=9 /
&MESH ID='mesh11', COLOR='COBALT', IJK=56,56,30, XB=1.,3.8,6.2,9.,1.5,3., MPI_PROCESS=10 /
&MESH ID='mesh12', COLOR='HOT PINK',IJK=48,56,30, XB=3.8,6.2,6.2,9.,1.5,3., MPI_PROCESS=11 /
&MESH ID='mesh13', COLOR='KELLY GREEN', IJK=56,56,30, XB=6.2,9.,6.2,9.,1.5,3., MPI_PROCESS=12 /
&MESH ID='mesh14', COLOR='TEAL', IJK=56,48,30, XB=1.,3.8,3.8,6.2,1.5,3., MPI_PROCESS=13 /
&MESH ID='mesh15', COLOR='YELLOW', IJK=56,48,30, XB=6.2,9.,3.8,6.2,1.5,3., MPI_PROCESS=14 /
&MESH ID='mesh16', COLOR='BROWN', IJK=56,56,30, XB=1.,3.8,1.,3.8,1.5,3., MPI_PROCESS=15 /
&MESH ID='mesh17', COLOR='CADET BLUE', IJK=48,56,30, XB=3.8,6.2,1.,3.8,1.5,3., MPI_PROCESS=16 /

&MISC VERBOSE=.TRUE./
&RADI RADIATION=.FALSE. /

&TIME T_END=300. /

&REAC FUEL = 'METHANE', SOOT_YIELD = 0.2, CO_YIELD = 0.10, RADIATIVE_FRACTION=0.3 /

&SURF ID='BURNER', HRRPUA=500., TAU_MF=0.01 /
&VENT XB=3.359,6.641,3.359,6.641,0.0,0.0, XYZ=5.0,5.0,0.0, RADIUS=1.641, SPREAD_RATE=0.00547, COLOR='RED', SURF_ID='BURNER' /


&VENT MB='XMIN', SURF_ID='OPEN' /  
&VENT MB='XMAX', SURF_ID='OPEN' /  
&VENT MB='YMIN', SURF_ID='OPEN' /  
&VENT MB='YMAX', SURF_ID='OPEN' / 
 
&SPEC ID='WATER VAPOR' /
&PART ID='my droplets', DIAMETER=1000., SPEC_ID='WATER VAPOR' /
&PROP ID='K-11', QUANTITY='SPRINKLER LINK TEMPERATURE', RTI=50., C_FACTOR=0.0, ACTIVATION_TEMPERATURE=6800., PART_ID='my droplets', FLOW_RATE=300.0, PARTICLE_VELOCITY=10., SMOKEVIEW_ID='sprinkler_pendent' /
&DEVC ID='Spr-1', XYZ=5.0,1.75,2.97, PROP_ID='K-11' /

&DEVC XYZ=5.0,1.75,2.97, QUANTITY='TEMPERATURE', ID='T-1'/
&DEVC XYZ=5.0,1.75,2.97, QUANTITY='VELOCITY', ID='U-1'/


&CTRL ID='kill', FUNCTION_TYPE='KILL', INPUT_ID='trigger' /
&CTRL ID='trigger', FUNCTION_TYPE='ALL', INPUT_ID='sprinklerActivated', 'delay' /
&CTRL ID='delay', FUNCTION_TYPE='TIME_DELAY', INPUT_ID='sprinklerActivated', DELAY=5. /
&CTRL ID='sprinklerActivated', FUNCTION_TYPE='AT_LEAST', N=1, INPUT_ID='Spr-1' /
&CTRL ID='Spr-1 Activated', FUNCTION_TYPE='AT_LEAST', N=1, INPUT_ID='Spr-1' /


&DUMP DT_HRR=1.0, DT_BNDF=1.0, DT_DEVC=1.0 /

&SLCF PBX=5., QUANTITY='TEMPERATURE' /
&SLCF PBZ=2.95, QUANTITY='TEMPERATURE' /
&SLCF PBX=5., QUANTITY='VELOCITY' /


&MISC RESTART=.FALSE.

&TAIL /
\end{lstlisting}

\subsection*{Raumhöhe gleich 6 m}
\begin{lstlisting}[emptylines=0,basicstyle=\tiny]
&HEAD CHID='OffenerRaum', TITLE='H=6 mit alt. Brandherd' /

&MESH ID='mesh1', COLOR='MAROON', IJK=28,28,30, XB=1.,3.8,6.2,9.,0.,3., MPI_PROCESS=0 /
&MESH ID='mesh2', COLOR='MELON',IJK=24,28,30, XB=3.8,6.2,6.2,9.,0.,3., MPI_PROCESS=1 /
&MESH ID='mesh3', COLOR='MINT', IJK=28,28,30, XB=6.2,9.,6.2,9.,0.,3., MPI_PROCESS=2 /
&MESH ID='mesh4', COLOR='OLIVE', IJK=28,24,30, XB=1.,3.8,3.8,6.2,0.,3., MPI_PROCESS=3 /
&MESH ID='mesh5', COLOR='ORCHID', IJK=28,24,30, XB=6.2,9.,3.8,6.2,0.,3., MPI_PROCESS=4 /
&MESH ID='mesh6', COLOR='SALMON', IJK=28,28,30, XB=1.,3.8,1.,3.8,0.,3., MPI_PROCESS=5 /
&MESH ID='mesh7', COLOR='STEEL BLUE', IJK=24,28,30, XB=3.8,6.2,1.,3.8,0.,3., MPI_PROCESS=6 /
&MESH ID='mesh8', COLOR='FLESH', IJK=28,28,30, XB=6.2,9.,1.,3.8,0.,3., MPI_PROCESS=7 /
&MESH ID='mesh9', COLOR='CYAN', IJK=24,24,45, XB=3.8,6.2,3.8,6.2,0.,4.5, MPI_PROCESS=8 /
&MESH ID='mesh10', COLOR='BLUE', IJK=24,24,15, XB=3.8,6.2,3.8,6.2,4.5,6., MPI_PROCESS=9 /
&MESH ID='mesh11', COLOR='CHOCOLATE', IJK=28,28,30, XB=1.,3.8,6.2,9.,3.,6., MPI_PROCESS=10 /
&MESH ID='mesh12', COLOR='COBALT',IJK=24,28,30, XB=3.8,6.2,6.2,9.,3.,6., MPI_PROCESS=11 /
&MESH ID='mesh13', COLOR='HOT PINK', IJK=28,28,30, XB=6.2,9.,6.2,9.,3.,6., MPI_PROCESS=12 /
&MESH ID='mesh14', COLOR='KELLY GREEN', IJK=28,24,30, XB=1.,3.8,3.8,6.2,3.,6., MPI_PROCESS=13 /
&MESH ID='mesh15', COLOR='TEAL', IJK=28,24,30, XB=6.2,9.,3.8,6.2,3.,6., MPI_PROCESS=14 /
&MESH ID='mesh16', COLOR='YELLOW', IJK=28,28,30, XB=1.,3.8,1.,3.8,3.,6., MPI_PROCESS=15 /
&MESH ID='mesh17', COLOR='BROWN', IJK=24,28,30, XB=3.8,6.2,1.,3.8,3.,6., MPI_PROCESS=16 /
&MESH ID='mesh18', COLOR='CADET BLUE', IJK=28,28,30, XB=6.2,9.,1.,3.8,3.,6., MPI_PROCESS=17 /

&MISC VERBOSE=.TRUE./
&RADI RADIATION=.FALSE. /

&TIME T_END=300. /

&DUMP DT_RESTART=5. /


&REAC FUEL = 'METHANE', SOOT_YIELD = 0.2, CO_YIELD = 0.10, RADIATIVE_FRACTION=0.3 /

&SURF ID='BURNER', HRRPUA=500., TAU_MF=0.01 /
&VENT XB=3.359,6.641,3.359,6.641,0.0,0.0, XYZ=5.0,5.0,0.0, RADIUS=1.641, SPREAD_RATE=0.00547, COLOR='RED', SURF_ID='BURNER' /


&VENT MB='XMIN', SURF_ID='OPEN' /  
&VENT MB='XMAX', SURF_ID='OPEN' /  
&VENT MB='YMIN', SURF_ID='OPEN' /  
&VENT MB='YMAX', SURF_ID='OPEN' / 

 
&SPEC ID='WATER VAPOR' /
&PART ID='my droplets', DIAMETER=1000., SPEC_ID='WATER VAPOR' /
&PROP ID='K-11', QUANTITY='SPRINKLER LINK TEMPERATURE', RTI=50., C_FACTOR=0.0, ACTIVATION_TEMPERATURE=6800., PART_ID='my droplets', FLOW_RATE=300.0, PARTICLE_VELOCITY=10., SMOKEVIEW_ID='sprinkler_pendent' /
&DEVC ID='Spr-1', XYZ=5.0,1.75,5.97, PROP_ID='K-11' /

&DEVC XYZ=5.0,1.75,5.97, QUANTITY='TEMPERATURE', ID='T-1'/
&DEVC XYZ=5.0,1.75,5.97, QUANTITY='VELOCITY', ID='U-1'/


&CTRL ID='kill', FUNCTION_TYPE='KILL', INPUT_ID='delay' /
&CTRL ID='delay', FUNCTION_TYPE='TIME_DELAY', INPUT_ID='Spr-1', DELAY=5. /




&DUMP DT_HRR=1.0, DT_BNDF=1.0, DT_DEVC=1.0 /

&SLCF PBX=5., QUANTITY='TEMPERATURE' /
&SLCF PBZ=2.95, QUANTITY='TEMPERATURE' /
&SLCF PBX=5., QUANTITY='VELOCITY' /


&MISC RESTART=.FALSE.

&TAIL /
\end{lstlisting}

\subsection*{Raumhöhe gleich 8 m}
\begin{lstlisting}[emptylines=0,basicstyle=\tiny]
&HEAD CHID='OffenerRaum', TITLE='H=8 mit alt. Brandherd' /

&MESH ID='mesh1', COLOR='MAROON', IJK=28,28,40, XB=1.,3.8,6.2,9.,0.,4., MPI_PROCESS=0 /
&MESH ID='mesh2', COLOR='MELON',IJK=24,28,40, XB=3.8,6.2,6.2,9.,0.,4., MPI_PROCESS=1 /
&MESH ID='mesh3', COLOR='MINT', IJK=28,28,40, XB=6.2,9.,6.2,9.,0.,4., MPI_PROCESS=2 /
&MESH ID='mesh4', COLOR='OLIVE', IJK=28,24,40, XB=1.,3.8,3.8,6.2,0.,4., MPI_PROCESS=3 /
&MESH ID='mesh5', COLOR='ORCHID', IJK=28,24,40, XB=6.2,9.,3.8,6.2,0.,4., MPI_PROCESS=4 /
&MESH ID='mesh6', COLOR='SALMON', IJK=28,28,40, XB=1.,3.8,1.,3.8,0.,4., MPI_PROCESS=5 /
&MESH ID='mesh7', COLOR='STEEL BLUE', IJK=24,28,40, XB=3.8,6.2,1.,3.8,0.,4., MPI_PROCESS=6 /
&MESH ID='mesh8', COLOR='FLESH', IJK=28,28,40, XB=6.2,9.,1.,3.8,0.,4., MPI_PROCESS=7 /
&MESH ID='mesh9', COLOR='CYAN', IJK=24,24,45, XB=3.8,6.2,3.8,6.2,0.,4.5, MPI_PROCESS=8 /
&MESH ID='mesh10', COLOR='BLUE', IJK=24,24,35, XB=3.8,6.2,3.8,6.2,4.5,8., MPI_PROCESS=9 /
&MESH ID='mesh11', COLOR='CHOCOLATE', IJK=28,28,40, XB=1.,3.8,6.2,9.,4.,8., MPI_PROCESS=10 /
&MESH ID='mesh12', COLOR='COBALT',IJK=24,28,40, XB=3.8,6.2,6.2,9.,4.,8., MPI_PROCESS=11 /
&MESH ID='mesh13', COLOR='HOT PINK', IJK=28,28,40, XB=6.2,9.,6.2,9.,4.,8., MPI_PROCESS=12 /
&MESH ID='mesh14', COLOR='KELLY GREEN', IJK=28,24,40, XB=1.,3.8,3.8,6.2,4.,8., MPI_PROCESS=13 /
&MESH ID='mesh15', COLOR='TEAL', IJK=28,24,40, XB=6.2,9.,3.8,6.2,4.,8., MPI_PROCESS=14 /
&MESH ID='mesh16', COLOR='YELLOW', IJK=28,28,40, XB=1.,3.8,1.,3.8,4.,8., MPI_PROCESS=15 /
&MESH ID='mesh17', COLOR='BROWN', IJK=24,28,40, XB=3.8,6.2,1.,3.8,4.,8., MPI_PROCESS=16 /
&MESH ID='mesh18', COLOR='CADET BLUE', IJK=28,28,40, XB=6.2,9.,1.,3.8,4.,8., MPI_PROCESS=17 /



&MISC VERBOSE=.TRUE./
&RADI RADIATION=.FALSE. /

&TIME T_END=300. /

&DUMP DT_RESTART=5. /


&REAC FUEL = 'METHANE', SOOT_YIELD = 0.2, CO_YIELD = 0.10, RADIATIVE_FRACTION=0.3 /

&SURF ID='BURNER', HRRPUA=500., TAU_MF=0.01 /
&VENT XB=3.359,6.641,3.359,6.641,0.0,0.0, XYZ=5.0,5.0,0.0, RADIUS=1.641, SPREAD_RATE=0.00547, COLOR='RED', SURF_ID='BURNER' /



&VENT MB='XMIN', SURF_ID='OPEN' /  
&VENT MB='XMAX', SURF_ID='OPEN' /  
&VENT MB='YMIN', SURF_ID='OPEN' /  
&VENT MB='YMAX', SURF_ID='OPEN' / 

 
&SPEC ID='WATER VAPOR' /
&PART ID='my droplets', DIAMETER=1000., SPEC_ID='WATER VAPOR' /
&PROP ID='K-11', QUANTITY='SPRINKLER LINK TEMPERATURE', RTI=50., C_FACTOR=0., ACTIVATION_TEMPERATURE=6800., PART_ID='my droplets', FLOW_RATE=300.0, PARTICLE_VELOCITY=10., SMOKEVIEW_ID='sprinkler_pendent' /
&DEVC ID='Spr-1', XYZ=5.0,1.75,7.97, PROP_ID='K-11' /

&DEVC XYZ=5.0,1.75,7.97, QUANTITY='TEMPERATURE', ID='T-1'/
&DEVC XYZ=5.0,1.75,7.97, QUANTITY='VELOCITY', ID='U-1'/


&CTRL ID='kill', FUNCTION_TYPE='KILL', INPUT_ID='trigger' /
&CTRL ID='trigger', FUNCTION_TYPE='ALL', INPUT_ID='sprinklerActivated', 'delay' /
&CTRL ID='delay', FUNCTION_TYPE='TIME_DELAY', INPUT_ID='sprinklerActivated', DELAY=5. /
&CTRL ID='sprinklerActivated', FUNCTION_TYPE='AT_LEAST', N=1, INPUT_ID='Spr-1' /
&CTRL ID='Spr-1 Activated', FUNCTION_TYPE='AT_LEAST', N=1, INPUT_ID='Spr-1' /



&DUMP DT_HRR=1.0, DT_BNDF=1.0, DT_DEVC=1.0 /

&SLCF PBX=5., QUANTITY='TEMPERATURE' /
&SLCF PBZ=2.95, QUANTITY='TEMPERATURE' /
&SLCF PBX=5., QUANTITY='VELOCITY' /


&MISC RESTART=.FALSE.

&TAIL /
\end{lstlisting}


%%%%%%%%%%%%%%%%%%%%%%%%%%%%%%%%%%%%%%%%%%%%%%%%%%%%%%%%%%%%%%%%%%%%%%%%
\section{C-Faktor Untersuchung}
Für die Untersuchung des C-Faktors variiert dieser in den nachfolgenden Dateien in Zeile 32 zwischen 0.0/0.5/1.0 und 1.5.
\subsection*{Raumhöhe gleich 3~m}
\subsubsection{$\alpha$ = 0,012~kW/s²}
\begin{lstlisting}[emptylines=0, basicstyle=\tiny]
    
&HEAD CHID='OffenerRaum', TITLE='H=3m, a=0,012, C=0' /

&MESH ID='mesh1', COLOR='MAROON', IJK=56,56,30, XB=1.,3.8,6.2,9.,0.,1.5, MPI_PROCESS=0 /
&MESH ID='mesh2', COLOR='MELON',IJK=48,56,30, XB=3.8,6.2,6.2,9.,0.,1.5, MPI_PROCESS=1 /
&MESH ID='mesh3', COLOR='MINT', IJK=56,56,30, XB=6.2,9.,6.2,9.,0.,1.5, MPI_PROCESS=2 /
&MESH ID='mesh4', COLOR='OLIVE', IJK=56,48,30, XB=1.,3.8,3.8,6.2,0.,1.5, MPI_PROCESS=3 /
&MESH ID='mesh5', COLOR='ORCHID', IJK=56,48,30, XB=6.2,9.,3.8,6.2,0.,1.5, MPI_PROCESS=4 /
&MESH ID='mesh6', COLOR='SALMON', IJK=56,56,30, XB=1.,3.8,1.,3.8,0.,1.5, MPI_PROCESS=5 /
&MESH ID='mesh7', COLOR='STEEL BLUE', IJK=48,56,30, XB=3.8,6.2,1.,3.8,0.,1.5, MPI_PROCESS=6 /
&MESH ID='mesh8', COLOR='FLESH', IJK=56,56,30, XB=6.2,9.,1.,3.8,0.,1.5, MPI_PROCESS=7 /
&MESH ID='mesh9', COLOR='BLUE', IJK=48,48,60, XB=3.8,6.2,3.8,6.2,0.,3., MPI_PROCESS=8 /
&MESH ID='mesh10', COLOR='CHOCOLATE', IJK=56,56,30, XB=6.2,9.,1.,3.8,1.5,3., MPI_PROCESS=9 /
&MESH ID='mesh11', COLOR='COBALT', IJK=56,56,30, XB=1.,3.8,6.2,9.,1.5,3., MPI_PROCESS=10 /
&MESH ID='mesh12', COLOR='HOT PINK',IJK=48,56,30, XB=3.8,6.2,6.2,9.,1.5,3., MPI_PROCESS=11 /
&MESH ID='mesh13', COLOR='KELLY GREEN', IJK=56,56,30, XB=6.2,9.,6.2,9.,1.5,3., MPI_PROCESS=12 /
&MESH ID='mesh14', COLOR='TEAL', IJK=56,48,30, XB=1.,3.8,3.8,6.2,1.5,3., MPI_PROCESS=13 /
&MESH ID='mesh15', COLOR='YELLOW', IJK=56,48,30, XB=6.2,9.,3.8,6.2,1.5,3., MPI_PROCESS=14 /
&MESH ID='mesh16', COLOR='BROWN', IJK=56,56,30, XB=1.,3.8,1.,3.8,1.5,3., MPI_PROCESS=15 /
&MESH ID='mesh17', COLOR='CADET BLUE', IJK=48,56,30, XB=3.8,6.2,1.,3.8,1.5,3., MPI_PROCESS=16 /

&MISC VERBOSE=.TRUE./
&RADI RADIATION=.FALSE. /

&TIME T_END=600. /

&DUMP DT_RESTART=5. /

&REAC FUEL = 'METHANE', SOOT_YIELD = 0.2, CO_YIELD = 0.10, RADIATIVE_FRACTION=0.3 /

&SURF ID='BURNER', HRRPUA=2444.6, TAU_MF=0.01 /
&VENT XB=4.25,5.75,4.25,5.75,0.00,0.00, XYZ=5.0,5.0,0.00, RADIUS=0.75, SPREAD_RATE=0.00125, COLOR='RED', SURF_ID='BURNER' /


&VENT MB='XMIN', SURF_ID='OPEN' /  
&VENT MB='XMAX', SURF_ID='OPEN' /  
&VENT MB='YMIN', SURF_ID='OPEN' /  
&VENT MB='YMAX', SURF_ID='OPEN' / 
 
&SPEC ID='WATER VAPOR' /
&PART ID='my droplets', DIAMETER=1000., SPEC_ID='WATER VAPOR' /
&PROP ID='K-11', QUANTITY='SPRINKLER LINK TEMPERATURE', RTI=50., C_FACTOR=0.0, ACTIVATION_TEMPERATURE=68., PART_ID='my droplets', FLOW_RATE=300.0, PARTICLE_VELOCITY=10., SMOKEVIEW_ID='sprinkler_pendent' /
&DEVC ID='Spr-1', XYZ=5.0,1.75,2.97, PROP_ID='K-11' /

&DEVC XYZ=5.0,1.75,2.97, QUANTITY='TEMPERATURE', ID='T-1'/
&DEVC XYZ=5.0,1.75,2.97, QUANTITY='VELOCITY', ID='U-1'/


&CTRL ID='kill', FUNCTION_TYPE='KILL', INPUT_ID='trigger' /
&CTRL ID='trigger', FUNCTION_TYPE='ALL', INPUT_ID='sprinklerActivated', 'delay' /
&CTRL ID='delay', FUNCTION_TYPE='TIME_DELAY', INPUT_ID='sprinklerActivated', DELAY=5. /
&CTRL ID='sprinklerActivated', FUNCTION_TYPE='AT_LEAST', N=1, INPUT_ID='Spr-1' /
&CTRL ID='Spr-1 Activated', FUNCTION_TYPE='AT_LEAST', N=1, INPUT_ID='Spr-1' /


&DUMP DT_HRR=1.0, DT_BNDF=1.0, DT_DEVC=1.0 /

&SLCF PBX=5., QUANTITY='TEMPERATURE' /
&SLCF PBZ=2.95, QUANTITY='TEMPERATURE' /
&SLCF PBX=5., QUANTITY='VELOCITY' /



&TAIL /
\end{lstlisting}

\subsubsection{$\alpha$ = 0,047~kW/s²}
\begin{lstlisting}[
    emptylines=0,basicstyle=\tiny]
&HEAD CHID='OffenerRaum', TITLE='H=3m, a=0,047, C=0' /

&MESH ID='mesh1', COLOR='MAROON', IJK=56,56,30, XB=1.,3.8,6.2,9.,0.,1.5, MPI_PROCESS=0 /
&MESH ID='mesh2', COLOR='MELON',IJK=48,56,30, XB=3.8,6.2,6.2,9.,0.,1.5, MPI_PROCESS=1 /
&MESH ID='mesh3', COLOR='MINT', IJK=56,56,30, XB=6.2,9.,6.2,9.,0.,1.5, MPI_PROCESS=2 /
&MESH ID='mesh4', COLOR='OLIVE', IJK=56,48,30, XB=1.,3.8,3.8,6.2,0.,1.5, MPI_PROCESS=3 /
&MESH ID='mesh5', COLOR='ORCHID', IJK=56,48,30, XB=6.2,9.,3.8,6.2,0.,1.5, MPI_PROCESS=4 /
&MESH ID='mesh6', COLOR='SALMON', IJK=56,56,30, XB=1.,3.8,1.,3.8,0.,1.5, MPI_PROCESS=5 /
&MESH ID='mesh7', COLOR='STEEL BLUE', IJK=48,56,30, XB=3.8,6.2,1.,3.8,0.,1.5, MPI_PROCESS=6 /
&MESH ID='mesh8', COLOR='FLESH', IJK=56,56,30, XB=6.2,9.,1.,3.8,0.,1.5, MPI_PROCESS=7 /
&MESH ID='mesh9', COLOR='BLUE', IJK=48,48,60, XB=3.8,6.2,3.8,6.2,0.,3., MPI_PROCESS=8 /
&MESH ID='mesh10', COLOR='CHOCOLATE', IJK=56,56,30, XB=6.2,9.,1.,3.8,1.5,3., MPI_PROCESS=9 /
&MESH ID='mesh11', COLOR='COBALT', IJK=56,56,30, XB=1.,3.8,6.2,9.,1.5,3., MPI_PROCESS=10 /
&MESH ID='mesh12', COLOR='HOT PINK',IJK=48,56,30, XB=3.8,6.2,6.2,9.,1.5,3., MPI_PROCESS=11 /
&MESH ID='mesh13', COLOR='KELLY GREEN', IJK=56,56,30, XB=6.2,9.,6.2,9.,1.5,3., MPI_PROCESS=12 /
&MESH ID='mesh14', COLOR='TEAL', IJK=56,48,30, XB=1.,3.8,3.8,6.2,1.5,3., MPI_PROCESS=13 /
&MESH ID='mesh15', COLOR='YELLOW', IJK=56,48,30, XB=6.2,9.,3.8,6.2,1.5,3., MPI_PROCESS=14 /
&MESH ID='mesh16', COLOR='BROWN', IJK=56,56,30, XB=1.,3.8,1.,3.8,1.5,3., MPI_PROCESS=15 /
&MESH ID='mesh17', COLOR='CADET BLUE', IJK=48,56,30, XB=3.8,6.2,1.,3.8,1.5,3., MPI_PROCESS=16 /

&MISC VERBOSE=.TRUE./
&RADI RADIATION=.FALSE. /

&TIME T_END=300. /

&DUMP DT_RESTART=5. /

&REAC FUEL = 'METHANE', SOOT_YIELD = 0.2, CO_YIELD = 0.10, RADIATIVE_FRACTION=0.3 /

&SURF ID='BURNER', HRRPUA=5385.8, TAU_MF=0.01 /
&VENT XB=4.5,5.5,4.5,5.5,0.0,0.0, XYZ=5.0,5.0,0.0, RADIUS=0.5, SPREAD_RATE=0.001666, COLOR='RED', SURF_ID='BURNER' /


&VENT MB='XMIN', SURF_ID='OPEN' /  
&VENT MB='XMAX', SURF_ID='OPEN' /  
&VENT MB='YMIN', SURF_ID='OPEN' /  
&VENT MB='YMAX', SURF_ID='OPEN' / 
 
&SPEC ID='WATER VAPOR' /
&PART ID='my droplets', DIAMETER=1000., SPEC_ID='WATER VAPOR' /
&PROP ID='K-11', QUANTITY='SPRINKLER LINK TEMPERATURE', RTI=50., C_FACTOR=0.0, ACTIVATION_TEMPERATURE=68., PART_ID='my droplets', FLOW_RATE=300.0, PARTICLE_VELOCITY=10., SMOKEVIEW_ID='sprinkler_pendent' /
&DEVC ID='Spr-1', XYZ=5.0,1.75,2.97, PROP_ID='K-11' /

&DEVC XYZ=5.0,1.75,2.97, QUANTITY='TEMPERATURE', ID='T-1'/
&DEVC XYZ=5.0,1.75,2.97, QUANTITY='VELOCITY', ID='U-1'/


&CTRL ID='kill', FUNCTION_TYPE='KILL', INPUT_ID='trigger' /
&CTRL ID='trigger', FUNCTION_TYPE='ALL', INPUT_ID='sprinklerActivated', 'delay' /
&CTRL ID='delay', FUNCTION_TYPE='TIME_DELAY', INPUT_ID='sprinklerActivated', DELAY=5. /
&CTRL ID='sprinklerActivated', FUNCTION_TYPE='AT_LEAST', N=1, INPUT_ID='Spr-1' /
&CTRL ID='Spr-1 Activated', FUNCTION_TYPE='AT_LEAST', N=1, INPUT_ID='Spr-1' /


&DUMP DT_HRR=1.0, DT_BNDF=1.0, DT_DEVC=1.0 /

&SLCF PBX=5., QUANTITY='TEMPERATURE' /
&SLCF PBZ=2.95, QUANTITY='TEMPERATURE' /
&SLCF PBX=5., QUANTITY='VELOCITY' /


&MISC RESTART=.FALSE.

&TAIL /

    
\end{lstlisting}
\subsubsection{$\alpha$ = 0,188~kW/s²}
\begin{lstlisting}[
    emptylines=0,basicstyle=\tiny]
&HEAD CHID='OffenerRaum', TITLE='H=3m, a=0,188, C=0' /

&MESH ID='mesh1', COLOR='MAROON', IJK=56,56,30, XB=1.,3.8,6.2,9.,0.,1.5, MPI_PROCESS=0 /
&MESH ID='mesh2', COLOR='MELON',IJK=48,56,30, XB=3.8,6.2,6.2,9.,0.,1.5, MPI_PROCESS=1 /
&MESH ID='mesh3', COLOR='MINT', IJK=56,56,30, XB=6.2,9.,6.2,9.,0.,1.5, MPI_PROCESS=2 /
&MESH ID='mesh4', COLOR='OLIVE', IJK=56,48,30, XB=1.,3.8,3.8,6.2,0.,1.5, MPI_PROCESS=3 /
&MESH ID='mesh5', COLOR='ORCHID', IJK=56,48,30, XB=6.2,9.,3.8,6.2,0.,1.5, MPI_PROCESS=4 /
&MESH ID='mesh6', COLOR='SALMON', IJK=56,56,30, XB=1.,3.8,1.,3.8,0.,1.5, MPI_PROCESS=5 /
&MESH ID='mesh7', COLOR='STEEL BLUE', IJK=48,56,30, XB=3.8,6.2,1.,3.8,0.,1.5, MPI_PROCESS=6 /
&MESH ID='mesh8', COLOR='FLESH', IJK=56,56,30, XB=6.2,9.,1.,3.8,0.,1.5, MPI_PROCESS=7 /
&MESH ID='mesh9', COLOR='BLUE', IJK=48,48,60, XB=3.8,6.2,3.8,6.2,0.,3., MPI_PROCESS=8 /
&MESH ID='mesh10', COLOR='CHOCOLATE', IJK=56,56,30, XB=6.2,9.,1.,3.8,1.5,3., MPI_PROCESS=9 /
&MESH ID='mesh11', COLOR='COBALT', IJK=56,56,30, XB=1.,3.8,6.2,9.,1.5,3., MPI_PROCESS=10 /
&MESH ID='mesh12', COLOR='HOT PINK',IJK=48,56,30, XB=3.8,6.2,6.2,9.,1.5,3., MPI_PROCESS=11 /
&MESH ID='mesh13', COLOR='KELLY GREEN', IJK=56,56,30, XB=6.2,9.,6.2,9.,1.5,3., MPI_PROCESS=12 /
&MESH ID='mesh14', COLOR='TEAL', IJK=56,48,30, XB=1.,3.8,3.8,6.2,1.5,3., MPI_PROCESS=13 /
&MESH ID='mesh15', COLOR='YELLOW', IJK=56,48,30, XB=6.2,9.,3.8,6.2,1.5,3., MPI_PROCESS=14 /
&MESH ID='mesh16', COLOR='BROWN', IJK=56,56,30, XB=1.,3.8,1.,3.8,1.5,3., MPI_PROCESS=15 /
&MESH ID='mesh17', COLOR='CADET BLUE', IJK=48,56,30, XB=3.8,6.2,1.,3.8,1.5,3., MPI_PROCESS=16 /
&MISC VERBOSE=.TRUE./
&RADI RADIATION=.FALSE. /

&TIME T_END=300. /

&DUMP DT_RESTART=5. /

&REAC FUEL = 'METHANE', SOOT_YIELD = 0.2, CO_YIELD = 0.10, RADIATIVE_FRACTION=0.3 /

&SURF ID='BURNER', HRRPUA=21543., TAU_MF=0.01 /
&VENT XB=4.5,5.5,4.5,5.5,0.00,0.00, XYZ=5.0,5.0,0.00, RADIUS=0.5, SPREAD_RATE=0.001666, COLOR='RED', SURF_ID='BURNER' /


&VENT MB='XMIN', SURF_ID='OPEN' /  
&VENT MB='XMAX', SURF_ID='OPEN' /  
&VENT MB='YMIN', SURF_ID='OPEN' /  
&VENT MB='YMAX', SURF_ID='OPEN' / 
 
&SPEC ID='WATER VAPOR' /
&PART ID='my droplets', DIAMETER=1000., SPEC_ID='WATER VAPOR' /
&PROP ID='K-11', QUANTITY='SPRINKLER LINK TEMPERATURE', RTI=50., C_FACTOR=0.0, ACTIVATION_TEMPERATURE=68., PART_ID='my droplets', FLOW_RATE=300.0, PARTICLE_VELOCITY=10., SMOKEVIEW_ID='sprinkler_pendent' /
&DEVC ID='Spr-1', XYZ=5.0,1.75,2.97, PROP_ID='K-11' /

&DEVC XYZ=5.0,1.75,2.97, QUANTITY='TEMPERATURE', ID='T-1'/
&DEVC XYZ=5.0,1.75,2.97, QUANTITY='VELOCITY', ID='U-1'/


&CTRL ID='kill', FUNCTION_TYPE='KILL', INPUT_ID='trigger' /
&CTRL ID='trigger', FUNCTION_TYPE='ALL', INPUT_ID='sprinklerActivated', 'delay' /
&CTRL ID='delay', FUNCTION_TYPE='TIME_DELAY', INPUT_ID='sprinklerActivated', DELAY=5. /
&CTRL ID='sprinklerActivated', FUNCTION_TYPE='AT_LEAST', N=1, INPUT_ID='Spr-1' /
&CTRL ID='Spr-1 Activated', FUNCTION_TYPE='AT_LEAST', N=1, INPUT_ID='Spr-1' /


&DUMP DT_HRR=1.0, DT_BNDF=1.0, DT_DEVC=1.0 /

&SLCF PBX=5., QUANTITY='TEMPERATURE' /
&SLCF PBZ=2.95, QUANTITY='TEMPERATURE' /
&SLCF PBX=5., QUANTITY='VELOCITY' /



&TAIL /

    
\end{lstlisting}
\subsubsection{Verschiedene RTI}
\begin{lstlisting}[emptylines=0,basicstyle=\tiny]
&HEAD CHID='OffenerRaum', TITLE='RTI=27, H=3m, a=0,047, C=0' /

&MESH ID='mesh1', COLOR='MAROON', IJK=56,56,30, XB=1.,3.8,6.2,9.,0.,1.5, MPI_PROCESS=0 /
&MESH ID='mesh2', COLOR='MELON',IJK=48,56,30, XB=3.8,6.2,6.2,9.,0.,1.5, MPI_PROCESS=1 /
&MESH ID='mesh3', COLOR='MINT', IJK=56,56,30, XB=6.2,9.,6.2,9.,0.,1.5, MPI_PROCESS=2 /
&MESH ID='mesh4', COLOR='OLIVE', IJK=56,48,30, XB=1.,3.8,3.8,6.2,0.,1.5, MPI_PROCESS=3 /
&MESH ID='mesh5', COLOR='ORCHID', IJK=56,48,30, XB=6.2,9.,3.8,6.2,0.,1.5, MPI_PROCESS=4 /
&MESH ID='mesh6', COLOR='SALMON', IJK=56,56,30, XB=1.,3.8,1.,3.8,0.,1.5, MPI_PROCESS=5 /
&MESH ID='mesh7', COLOR='STEEL BLUE', IJK=48,56,30, XB=3.8,6.2,1.,3.8,0.,1.5, MPI_PROCESS=6 /
&MESH ID='mesh8', COLOR='FLESH', IJK=56,56,30, XB=6.2,9.,1.,3.8,0.,1.5, MPI_PROCESS=7 /
&MESH ID='mesh9', COLOR='BLUE', IJK=48,48,60, XB=3.8,6.2,3.8,6.2,0.,3., MPI_PROCESS=8 /
&MESH ID='mesh10', COLOR='CHOCOLATE', IJK=56,56,30, XB=6.2,9.,1.,3.8,1.5,3., MPI_PROCESS=9 /
&MESH ID='mesh11', COLOR='COBALT', IJK=56,56,30, XB=1.,3.8,6.2,9.,1.5,3., MPI_PROCESS=10 /
&MESH ID='mesh12', COLOR='HOT PINK',IJK=48,56,30, XB=3.8,6.2,6.2,9.,1.5,3., MPI_PROCESS=11 /
&MESH ID='mesh13', COLOR='KELLY GREEN', IJK=56,56,30, XB=6.2,9.,6.2,9.,1.5,3., MPI_PROCESS=12 /
&MESH ID='mesh14', COLOR='TEAL', IJK=56,48,30, XB=1.,3.8,3.8,6.2,1.5,3., MPI_PROCESS=13 /
&MESH ID='mesh15', COLOR='YELLOW', IJK=56,48,30, XB=6.2,9.,3.8,6.2,1.5,3., MPI_PROCESS=14 /
&MESH ID='mesh16', COLOR='BROWN', IJK=56,56,30, XB=1.,3.8,1.,3.8,1.5,3., MPI_PROCESS=15 /
&MESH ID='mesh17', COLOR='CADET BLUE', IJK=48,56,30, XB=3.8,6.2,1.,3.8,1.5,3., MPI_PROCESS=16 /

&MISC VERBOSE=.TRUE./
&RADI RADIATION=.FALSE. /

&TIME T_END=300. /

&DUMP DT_RESTART=5. /

&REAC FUEL = 'METHANE', SOOT_YIELD = 0.2, CO_YIELD = 0.10, RADIATIVE_FRACTION=0.3 /

&SURF ID='BURNER', HRRPUA=5385.8, TAU_MF=0.01 /
&VENT XB=4.5,5.5,4.5,5.5,0.0,0.0, XYZ=5.0,5.0,0.0, RADIUS=0.5, SPREAD_RATE=0.001666, COLOR='RED', SURF_ID='BURNER' /

&VENT MB='XMIN', SURF_ID='OPEN' /  
&VENT MB='XMAX', SURF_ID='OPEN' /  
&VENT MB='YMIN', SURF_ID='OPEN' /  
&VENT MB='YMAX', SURF_ID='OPEN' / 
 
&SPEC ID='WATER VAPOR' /
&PART ID='my droplets', DIAMETER=1000., SPEC_ID='WATER VAPOR' /
&PROP ID='K-11', QUANTITY='SPRINKLER LINK TEMPERATURE', RTI=27., C_FACTOR=0.0, ACTIVATION_TEMPERATURE=68., PART_ID='my droplets', FLOW_RATE=300.0, PARTICLE_VELOCITY=10., SMOKEVIEW_ID='sprinkler_pendent' /
&DEVC ID='Spr-1', XYZ=5.0,1.75,2.97, PROP_ID='K-11' /

&DEVC XYZ=5.0,1.75,2.97, QUANTITY='TEMPERATURE', ID='T-1'/
&DEVC XYZ=5.0,1.75,2.97, QUANTITY='VELOCITY', ID='U-1'/


&CTRL ID='kill', FUNCTION_TYPE='KILL', INPUT_ID='trigger' /
&CTRL ID='trigger', FUNCTION_TYPE='ALL', INPUT_ID='sprinklerActivated', 'delay' /
&CTRL ID='delay', FUNCTION_TYPE='TIME_DELAY', INPUT_ID='sprinklerActivated', DELAY=5. /
&CTRL ID='sprinklerActivated', FUNCTION_TYPE='AT_LEAST', N=1, INPUT_ID='Spr-1' /
&CTRL ID='Spr-1 Activated', FUNCTION_TYPE='AT_LEAST', N=1, INPUT_ID='Spr-1' /


&DUMP DT_HRR=1.0, DT_BNDF=1.0, DT_DEVC=1.0 /

&SLCF PBX=5., QUANTITY='TEMPERATURE' /
&SLCF PBZ=2.95, QUANTITY='TEMPERATURE' /
&SLCF PBX=5., QUANTITY='VELOCITY' /



&TAIL /

    
\end{lstlisting}

\subsection*{Raumhöhe gleich 6~m}
\begin{lstlisting}[emptylines=0,basicstyle=\tiny]
&HEAD CHID='OffenerRaum', TITLE='a=0,047, C=0, H=6' /

&MESH ID='mesh1', COLOR='MAROON', IJK=28,28,30, XB=1.,3.8,6.2,9.,0.,3., MPI_PROCESS=0 /
&MESH ID='mesh2', COLOR='MELON',IJK=24,28,30, XB=3.8,6.2,6.2,9.,0.,3., MPI_PROCESS=1 /
&MESH ID='mesh3', COLOR='MINT', IJK=28,28,30, XB=6.2,9.,6.2,9.,0.,3., MPI_PROCESS=2 /
&MESH ID='mesh4', COLOR='OLIVE', IJK=28,24,30, XB=1.,3.8,3.8,6.2,0.,3., MPI_PROCESS=3 /
&MESH ID='mesh5', COLOR='ORCHID', IJK=28,24,30, XB=6.2,9.,3.8,6.2,0.,3., MPI_PROCESS=4 /
&MESH ID='mesh6', COLOR='SALMON', IJK=28,28,30, XB=1.,3.8,1.,3.8,0.,3., MPI_PROCESS=5 /
&MESH ID='mesh7', COLOR='STEEL BLUE', IJK=24,28,30, XB=3.8,6.2,1.,3.8,0.,3., MPI_PROCESS=6 /
&MESH ID='mesh8', COLOR='FLESH', IJK=28,28,30, XB=6.2,9.,1.,3.8,0.,3., MPI_PROCESS=7 /
&MESH ID='mesh9', COLOR='CYAN', IJK=24,24,45, XB=3.8,6.2,3.8,6.2,0.,4.5, MPI_PROCESS=8 /
&MESH ID='mesh10', COLOR='BLUE', IJK=24,24,15, XB=3.8,6.2,3.8,6.2,4.5,6., MPI_PROCESS=9 /
&MESH ID='mesh11', COLOR='CHOCOLATE', IJK=28,28,30, XB=1.,3.8,6.2,9.,3.,6., MPI_PROCESS=10 /
&MESH ID='mesh12', COLOR='COBALT',IJK=24,28,30, XB=3.8,6.2,6.2,9.,3.,6., MPI_PROCESS=11 /
&MESH ID='mesh13', COLOR='HOT PINK', IJK=28,28,30, XB=6.2,9.,6.2,9.,3.,6., MPI_PROCESS=12 /
&MESH ID='mesh14', COLOR='KELLY GREEN', IJK=28,24,30, XB=1.,3.8,3.8,6.2,3.,6., MPI_PROCESS=13 /
&MESH ID='mesh15', COLOR='TEAL', IJK=28,24,30, XB=6.2,9.,3.8,6.2,3.,6., MPI_PROCESS=14 /
&MESH ID='mesh16', COLOR='YELLOW', IJK=28,28,30, XB=1.,3.8,1.,3.8,3.,6., MPI_PROCESS=15 /
&MESH ID='mesh17', COLOR='BROWN', IJK=24,28,30, XB=3.8,6.2,1.,3.8,3.,6., MPI_PROCESS=16 /
&MESH ID='mesh18', COLOR='CADET BLUE', IJK=28,28,30, XB=6.2,9.,1.,3.8,3.,6., MPI_PROCESS=17 /

&MISC VERBOSE=.TRUE./
&RADI RADIATION=.FALSE. /

&TIME T_END=300. /

&DUMP DT_RESTART=5. /


&REAC FUEL = 'METHANE', SOOT_YIELD = 0.2, CO_YIELD = 0.10, RADIATIVE_FRACTION=0.3 /

&SURF ID='BURNER', HRRPUA=5385.8, TAU_MF=0.01 /
&VENT XB=4.5,5.5,4.5,5.5,0.0,0.0, XYZ=5.0,5.0,0.0, RADIUS=0.5, SPREAD_RATE=0.001666, COLOR='RED', SURF_ID='BURNER' /


&VENT MB='XMIN', SURF_ID='OPEN' /  
&VENT MB='XMAX', SURF_ID='OPEN' /  
&VENT MB='YMIN', SURF_ID='OPEN' /  
&VENT MB='YMAX', SURF_ID='OPEN' / 

 
&SPEC ID='WATER VAPOR' /
&PART ID='my droplets', DIAMETER=1000., SPEC_ID='WATER VAPOR' /
&PROP ID='K-11', QUANTITY='SPRINKLER LINK TEMPERATURE', RTI=50., C_FACTOR=0.0, ACTIVATION_TEMPERATURE=68., PART_ID='my droplets', FLOW_RATE=300.0, PARTICLE_VELOCITY=10., SMOKEVIEW_ID='sprinkler_pendent' /
&DEVC ID='Spr-1', XYZ=5.0,1.75,5.97, PROP_ID='K-11' /

&DEVC XYZ=5.0,1.75,5.97, QUANTITY='TEMPERATURE', ID='T-1'/
&DEVC XYZ=5.0,1.75,5.97, QUANTITY='VELOCITY', ID='U-1'/


&CTRL ID='kill', FUNCTION_TYPE='KILL', INPUT_ID='delay' /
&CTRL ID='delay', FUNCTION_TYPE='TIME_DELAY', INPUT_ID='Spr-1', DELAY=5. /




&DUMP DT_HRR=1.0, DT_BNDF=1.0, DT_DEVC=1.0 /

&SLCF PBX=5., QUANTITY='TEMPERATURE' /
&SLCF PBZ=2.95, QUANTITY='TEMPERATURE' /
&SLCF PBX=5., QUANTITY='VELOCITY' /


&MISC RESTART=.FALSE.

&TAIL /

    
\end{lstlisting}

\subsection*{Raumhöhe gleich 8~m}
\begin{lstlisting}[emptylines=0,basicstyle=\tiny]
&HEAD CHID='OffenerRaum', TITLE='a=0,047, C=0, H=8' /

&MESH ID='mesh1', COLOR='MAROON', IJK=28,28,40, XB=1.,3.8,6.2,9.,0.,4., MPI_PROCESS=0 /
&MESH ID='mesh2', COLOR='MELON',IJK=24,28,40, XB=3.8,6.2,6.2,9.,0.,4., MPI_PROCESS=1 /
&MESH ID='mesh3', COLOR='MINT', IJK=28,28,40, XB=6.2,9.,6.2,9.,0.,4., MPI_PROCESS=2 /
&MESH ID='mesh4', COLOR='OLIVE', IJK=28,24,40, XB=1.,3.8,3.8,6.2,0.,4., MPI_PROCESS=3 /
&MESH ID='mesh5', COLOR='ORCHID', IJK=28,24,40, XB=6.2,9.,3.8,6.2,0.,4., MPI_PROCESS=4 /
&MESH ID='mesh6', COLOR='SALMON', IJK=28,28,40, XB=1.,3.8,1.,3.8,0.,4., MPI_PROCESS=5 /
&MESH ID='mesh7', COLOR='STEEL BLUE', IJK=24,28,40, XB=3.8,6.2,1.,3.8,0.,4., MPI_PROCESS=6 /
&MESH ID='mesh8', COLOR='FLESH', IJK=28,28,40, XB=6.2,9.,1.,3.8,0.,4.5, MPI_PROCESS=7 /
&MESH ID='mesh9', COLOR='CYAN', IJK=24,24,45, XB=3.8,6.2,3.8,6.2,0.,4.5, MPI_PROCESS=8 /
&MESH ID='mesh10', COLOR='BLUE', IJK=24,24,35, XB=3.8,6.2,3.8,6.2,4.5,8., MPI_PROCESS=9 /
&MESH ID='mesh11', COLOR='CHOCOLATE', IJK=28,28,40, XB=1.,3.8,6.2,9.,4.,8., MPI_PROCESS=10 /
&MESH ID='mesh12', COLOR='COBALT',IJK=24,28,40, XB=3.8,6.2,6.2,9.,4.,8., MPI_PROCESS=11 /
&MESH ID='mesh13', COLOR='HOT PINK', IJK=28,28,40, XB=6.2,9.,6.2,9.,4.,8., MPI_PROCESS=12 /
&MESH ID='mesh14', COLOR='KELLY GREEN', IJK=28,24,40, XB=1.,3.8,3.8,6.2,4.,8., MPI_PROCESS=13 /
&MESH ID='mesh15', COLOR='TEAL', IJK=28,24,40, XB=6.2,9.,3.8,6.2,4.,8., MPI_PROCESS=14 /
&MESH ID='mesh16', COLOR='YELLOW', IJK=28,28,40, XB=1.,3.8,1.,3.8,4.,8., MPI_PROCESS=15 /
&MESH ID='mesh17', COLOR='BROWN', IJK=24,28,40, XB=3.8,6.2,1.,3.8,4.,8., MPI_PROCESS=16 /
&MESH ID='mesh18', COLOR='CADET BLUE', IJK=28,28,40, XB=6.2,9.,1.,3.8,4.,8., MPI_PROCESS=17 /


&MISC VERBOSE=.TRUE./
&RADI RADIATION=.FALSE. /

&TIME T_END=400. /

&DUMP DT_RESTART=5. /


&REAC FUEL = 'METHANE', SOOT_YIELD = 0.2, CO_YIELD = 0.10, RADIATIVE_FRACTION=0.3 /

&SURF ID='BURNER', HRRPUA=4255.5, TAU_MF=0.01 /
&VENT XB=4.25,5.75,4.25,5.75,0.0,0.0, XYZ=5.0,5.0,0.0, RADIUS=0.75, SPREAD_RATE=0.001875, COLOR='RED', SURF_ID='BURNER' /


&VENT MB='XMIN', SURF_ID='OPEN' /  
&VENT MB='XMAX', SURF_ID='OPEN' /  
&VENT MB='YMIN', SURF_ID='OPEN' /  
&VENT MB='YMAX', SURF_ID='OPEN' / 

 
&SPEC ID='WATER VAPOR' /
&PART ID='my droplets', DIAMETER=1000., SPEC_ID='WATER VAPOR' /
&PROP ID='K-11', QUANTITY='SPRINKLER LINK TEMPERATURE', RTI=50., C_FACTOR=0., ACTIVATION_TEMPERATURE=68., PART_ID='my droplets', FLOW_RATE=300.0, PARTICLE_VELOCITY=10., SMOKEVIEW_ID='sprinkler_pendent' /
&DEVC ID='Spr-1', XYZ=5.0,1.75,7.97, PROP_ID='K-11' /

&DEVC XYZ=5.0,1.75,7.97, QUANTITY='TEMPERATURE', ID='T-1'/
&DEVC XYZ=5.0,1.75,7.97, QUANTITY='VELOCITY', ID='U-1'/


&CTRL ID='kill', FUNCTION_TYPE='KILL', INPUT_ID='trigger' /
&CTRL ID='trigger', FUNCTION_TYPE='ALL', INPUT_ID='sprinklerActivated', 'delay' /
&CTRL ID='delay', FUNCTION_TYPE='TIME_DELAY', INPUT_ID='sprinklerActivated', DELAY=5. /
&CTRL ID='sprinklerActivated', FUNCTION_TYPE='AT_LEAST', N=1, INPUT_ID='Spr-1' /
&CTRL ID='Spr-1 Activated', FUNCTION_TYPE='AT_LEAST', N=1, INPUT_ID='Spr-1' /



&DUMP DT_HRR=1.0, DT_BNDF=1.0, DT_DEVC=1.0 /

&SLCF PBX=5., QUANTITY='TEMPERATURE' /
&SLCF PBZ=2.95, QUANTITY='TEMPERATURE' /
&SLCF PBX=5., QUANTITY='VELOCITY' /


&MISC RESTART=.FALSE.

&TAIL /

    
\end{lstlisting}
%%%%%%%%%%%%%%%%%%%%%%%%%%%%%%%%%%%%%%%%%%%%%%%%
\section{Plot3D}
\begin{lstlisting}[emptylines=0,basicstyle=\tiny]
&HEAD CHID='OffenerRaum', TITLE='H=3m, a=0,047, C=0, Plot3d Test' /

&MESH ID='mesh1', COLOR='MAROON', IJK=56,56,30, XB=1.,3.8,6.2,9.,0.,1.5, MPI_PROCESS=0 /
&MESH ID='mesh2', COLOR='MELON',IJK=48,56,30, XB=3.8,6.2,6.2,9.,0.,1.5, MPI_PROCESS=1 /
&MESH ID='mesh3', COLOR='MINT', IJK=56,56,30, XB=6.2,9.,6.2,9.,0.,1.5, MPI_PROCESS=2 /
&MESH ID='mesh4', COLOR='OLIVE', IJK=56,48,30, XB=1.,3.8,3.8,6.2,0.,1.5, MPI_PROCESS=3 /
&MESH ID='mesh5', COLOR='ORCHID', IJK=56,48,30, XB=6.2,9.,3.8,6.2,0.,1.5, MPI_PROCESS=4 /
&MESH ID='mesh6', COLOR='SALMON', IJK=56,56,30, XB=1.,3.8,1.,3.8,0.,1.5, MPI_PROCESS=5 /
&MESH ID='mesh7', COLOR='STEEL BLUE', IJK=48,56,30, XB=3.8,6.2,1.,3.8,0.,1.5, MPI_PROCESS=6 /
&MESH ID='mesh8', COLOR='FLESH', IJK=56,56,30, XB=6.2,9.,1.,3.8,0.,1.5, MPI_PROCESS=7 /
&MESH ID='mesh9', COLOR='BLUE', IJK=48,48,60, XB=3.8,6.2,3.8,6.2,0.,3., MPI_PROCESS=8 /
&MESH ID='mesh10', COLOR='CHOCOLATE', IJK=56,56,30, XB=6.2,9.,1.,3.8,1.5,3., MPI_PROCESS=9 /
&MESH ID='mesh11', COLOR='COBALT', IJK=56,56,30, XB=1.,3.8,6.2,9.,1.5,3., MPI_PROCESS=10 /
&MESH ID='mesh12', COLOR='HOT PINK',IJK=48,56,30, XB=3.8,6.2,6.2,9.,1.5,3., MPI_PROCESS=11 /
&MESH ID='mesh13', COLOR='KELLY GREEN', IJK=56,56,30, XB=6.2,9.,6.2,9.,1.5,3., MPI_PROCESS=12 /
&MESH ID='mesh14', COLOR='TEAL', IJK=56,48,30, XB=1.,3.8,3.8,6.2,1.5,3., MPI_PROCESS=13 /
&MESH ID='mesh15', COLOR='YELLOW', IJK=56,48,30, XB=6.2,9.,3.8,6.2,1.5,3., MPI_PROCESS=14 /
&MESH ID='mesh16', COLOR='BROWN', IJK=56,56,30, XB=1.,3.8,1.,3.8,1.5,3., MPI_PROCESS=15 /
&MESH ID='mesh17', COLOR='CADET BLUE', IJK=48,56,30, XB=3.8,6.2,1.,3.8,1.5,3., MPI_PROCESS=16 /

&MISC VERBOSE=.TRUE./
&RADI RADIATION=.FALSE. /

&TIME T_END=300. /

&DUMP DT_RESTART=5. /

&REAC FUEL = 'METHANE', SOOT_YIELD = 0.2, CO_YIELD = 0.10, RADIATIVE_FRACTION=0.3 /

&SURF ID='BURNER', HRRPUA=5385.8, TAU_MF=0.01 /
&VENT XB=4.5,5.5,4.5,5.5,0.0,0.0, XYZ=5.0,5.0,0.0, RADIUS=0.5, SPREAD_RATE=0.001666, COLOR='RED', SURF_ID='BURNER' /


&VENT MB='XMIN', SURF_ID='OPEN' /  
&VENT MB='XMAX', SURF_ID='OPEN' /  
&VENT MB='YMIN', SURF_ID='OPEN' /  
&VENT MB='YMAX', SURF_ID='OPEN' / 
 
&SPEC ID='WATER VAPOR' /
&PART ID='my droplets', DIAMETER=1000., SPEC_ID='WATER VAPOR' /
&PROP ID='K-11', QUANTITY='SPRINKLER LINK TEMPERATURE', RTI=50., C_FACTOR=0.0, ACTIVATION_TEMPERATURE=68., PART_ID='my droplets', FLOW_RATE=300.0, PARTICLE_VELOCITY=10., SMOKEVIEW_ID='sprinkler_pendent' /
&DEVC ID='Spr-1', XYZ=5.0,1.75,2.97, PROP_ID='K-11' /

&DEVC XYZ=5.0,1.75,2.97, QUANTITY='TEMPERATURE', ID='T-1'/
&DEVC XYZ=5.0,1.75,2.97, QUANTITY='VELOCITY', ID='U-1'/


&CTRL ID='kill', FUNCTION_TYPE='KILL', INPUT_ID='trigger' /
&CTRL ID='trigger', FUNCTION_TYPE='ALL', INPUT_ID='sprinklerActivated', 'delay' /
&CTRL ID='delay', FUNCTION_TYPE='TIME_DELAY', INPUT_ID='sprinklerActivated', DELAY=5. /
&CTRL ID='sprinklerActivated', FUNCTION_TYPE='AT_LEAST', N=1, INPUT_ID='Spr-1' /
&CTRL ID='Spr-1 Activated', FUNCTION_TYPE='AT_LEAST', N=1, INPUT_ID='Spr-1' /


&DUMP DT_HRR=1.0, DT_BNDF=1.0, DT_DEVC=1.0, DT_PL3D=1.0 /
&DUMP PLOT3D_QUANTITY(1:5)='TEMPERATURE', 'U-VELOCITY', 'V-VELOCITY', 'W-VELOCITY', 'HRRPUV' /

&SLCF PBX=5., QUANTITY='TEMPERATURE' /
&SLCF PBZ=2.95, QUANTITY='TEMPERATURE' /
&SLCF PBX=5., QUANTITY='VELOCITY' /


&MISC RESTART=.FALSE.

&TAIL /


\end{lstlisting}
%%%%%%%%%%%%%%%%%%%%%%%%%%%%%%%%%%%%%%%%%%%%%%%%%%

\section{Raumhöhen}
\subsection*{Raumhöhe gleich 6}
\begin{lstlisting}[emptylines=0,basicstyle=\tiny]
&HEAD CHID='OffenerRaum', TITLE='H=6m bis 300s' /

&MESH ID='mesh1', COLOR='MAROON', IJK=28,28,30, XB=1.,3.8,6.2,9.,0.,3., MPI_PROCESS=0 /
&MESH ID='mesh2', COLOR='MELON',IJK=24,28,30, XB=3.8,6.2,6.2,9.,0.,3., MPI_PROCESS=1 /
&MESH ID='mesh3', COLOR='MINT', IJK=28,28,30, XB=6.2,9.,6.2,9.,0.,3., MPI_PROCESS=2 /
&MESH ID='mesh4', COLOR='OLIVE', IJK=28,24,30, XB=1.,3.8,3.8,6.2,0.,3., MPI_PROCESS=3 /
&MESH ID='mesh5', COLOR='ORCHID', IJK=28,24,30, XB=6.2,9.,3.8,6.2,0.,3., MPI_PROCESS=4 /
&MESH ID='mesh6', COLOR='SALMON', IJK=28,28,30, XB=1.,3.8,1.,3.8,0.,3., MPI_PROCESS=5 /
&MESH ID='mesh7', COLOR='STEEL BLUE', IJK=24,28,30, XB=3.8,6.2,1.,3.8,0.,3., MPI_PROCESS=6 /
&MESH ID='mesh8', COLOR='FLESH', IJK=28,28,30, XB=6.2,9.,1.,3.8,0.,3., MPI_PROCESS=7 /
&MESH ID='mesh9', COLOR='CYAN', IJK=24,24,45, XB=3.8,6.2,3.8,6.2,0.,4.5, MPI_PROCESS=8 /
&MESH ID='mesh10', COLOR='BLUE', IJK=24,24,15, XB=3.8,6.2,3.8,6.2,4.5,6., MPI_PROCESS=9 /
&MESH ID='mesh11', COLOR='CHOCOLATE', IJK=28,28,30, XB=1.,3.8,6.2,9.,3.,6., MPI_PROCESS=10 /
&MESH ID='mesh12', COLOR='COBALT',IJK=24,28,30, XB=3.8,6.2,6.2,9.,3.,6., MPI_PROCESS=11 /
&MESH ID='mesh13', COLOR='HOT PINK', IJK=28,28,30, XB=6.2,9.,6.2,9.,3.,6., MPI_PROCESS=12 /
&MESH ID='mesh14', COLOR='KELLY GREEN', IJK=28,24,30, XB=1.,3.8,3.8,6.2,3.,6., MPI_PROCESS=13 /
&MESH ID='mesh15', COLOR='TEAL', IJK=28,24,30, XB=6.2,9.,3.8,6.2,3.,6., MPI_PROCESS=14 /
&MESH ID='mesh16', COLOR='YELLOW', IJK=28,28,30, XB=1.,3.8,1.,3.8,3.,6., MPI_PROCESS=15 /
&MESH ID='mesh17', COLOR='BROWN', IJK=24,28,30, XB=3.8,6.2,1.,3.8,3.,6., MPI_PROCESS=16 /
&MESH ID='mesh18', COLOR='CADET BLUE', IJK=28,28,30, XB=6.2,9.,1.,3.8,3.,6., MPI_PROCESS=17 /

&MISC VERBOSE=.TRUE./
&RADI RADIATION=.FALSE. /

&TIME T_END=300. /


&REAC FUEL = 'METHANE', SOOT_YIELD = 0.2, CO_YIELD = 0.10, RADIATIVE_FRACTION=0.3 /

&SURF ID='BURNER', HRRPUA=5385.8, TAU_MF=0.01 /
&VENT XB=4.5,5.5,4.5,5.5,0.0,0.0, XYZ=5.0,5.0,0.0, RADIUS=0.5, SPREAD_RATE=0.001666, COLOR='RED', SURF_ID='BURNER' /


&VENT MB='XMIN', SURF_ID='OPEN' /  
&VENT MB='XMAX', SURF_ID='OPEN' /  
&VENT MB='YMIN', SURF_ID='OPEN' /  
&VENT MB='YMAX', SURF_ID='OPEN' / 

 
&SPEC ID='WATER VAPOR' /
&PART ID='my droplets', DIAMETER=1000., SPEC_ID='WATER VAPOR' /
&PROP ID='K-11', QUANTITY='SPRINKLER LINK TEMPERATURE', RTI=50., C_FACTOR=0.0, ACTIVATION_TEMPERATURE=6800., PART_ID='my droplets', FLOW_RATE=300.0, PARTICLE_VELOCITY=10., SMOKEVIEW_ID='sprinkler_pendent' /
&DEVC ID='Spr-1', XYZ=5.0,1.75,5.97, PROP_ID='K-11' /

&DEVC XYZ=5.0,1.75,5.97, QUANTITY='TEMPERATURE', ID='T-1'/
&DEVC XYZ=5.0,1.75,5.97, QUANTITY='VELOCITY', ID='U-1'/


&CTRL ID='kill', FUNCTION_TYPE='KILL', INPUT_ID='delay' /
&CTRL ID='delay', FUNCTION_TYPE='TIME_DELAY', INPUT_ID='Spr-1', DELAY=5. /


&DUMP DT_HRR=1.0, DT_BNDF=1.0, DT_DEVC=1.0 /

&SLCF PBX=5., QUANTITY='TEMPERATURE' /
&SLCF PBZ=2.95, QUANTITY='TEMPERATURE' /
&SLCF PBX=5., QUANTITY='VELOCITY' /


&TAIL /


\end{lstlisting}

\subsection*{Raumhöhe gleich 8}
\begin{lstlisting}[emptylines=0,basicstyle=\tiny]
&HEAD CHID='OffenerRaum', TITLE='H=8m bis 300s' /

&MESH ID='mesh1', COLOR='MAROON', IJK=28,28,40, XB=1.,3.8,6.2,9.,0.,4., MPI_PROCESS=0 /
&MESH ID='mesh2', COLOR='MELON',IJK=24,28,40, XB=3.8,6.2,6.2,9.,0.,4., MPI_PROCESS=1 /
&MESH ID='mesh3', COLOR='MINT', IJK=28,28,40, XB=6.2,9.,6.2,9.,0.,4., MPI_PROCESS=2 /
&MESH ID='mesh4', COLOR='OLIVE', IJK=28,24,40, XB=1.,3.8,3.8,6.2,0.,4., MPI_PROCESS=3 /
&MESH ID='mesh5', COLOR='ORCHID', IJK=28,24,40, XB=6.2,9.,3.8,6.2,0.,4., MPI_PROCESS=4 /
&MESH ID='mesh6', COLOR='SALMON', IJK=28,28,40, XB=1.,3.8,1.,3.8,0.,4., MPI_PROCESS=5 /
&MESH ID='mesh7', COLOR='STEEL BLUE', IJK=24,28,40, XB=3.8,6.2,1.,3.8,0.,4., MPI_PROCESS=6 /
&MESH ID='mesh8', COLOR='FLESH', IJK=28,28,40, XB=6.2,9.,1.,3.8,0.,4., MPI_PROCESS=7 /
&MESH ID='mesh9', COLOR='CYAN', IJK=24,24,45, XB=3.8,6.2,3.8,6.2,0.,4.5, MPI_PROCESS=8 /
&MESH ID='mesh10', COLOR='BLUE', IJK=24,24,35, XB=3.8,6.2,3.8,6.2,4.5,8., MPI_PROCESS=9 /
&MESH ID='mesh11', COLOR='CHOCOLATE', IJK=28,28,40, XB=1.,3.8,6.2,9.,4.,8., MPI_PROCESS=10 /
&MESH ID='mesh12', COLOR='COBALT',IJK=24,28,40, XB=3.8,6.2,6.2,9.,4.,8., MPI_PROCESS=11 /
&MESH ID='mesh13', COLOR='HOT PINK', IJK=28,28,40, XB=6.2,9.,6.2,9.,4.,8., MPI_PROCESS=12 /
&MESH ID='mesh14', COLOR='KELLY GREEN', IJK=28,24,40, XB=1.,3.8,3.8,6.2,4.,8., MPI_PROCESS=13 /
&MESH ID='mesh15', COLOR='TEAL', IJK=28,24,40, XB=6.2,9.,3.8,6.2,4.,8., MPI_PROCESS=14 /
&MESH ID='mesh16', COLOR='YELLOW', IJK=28,28,40, XB=1.,3.8,1.,3.8,4.,8., MPI_PROCESS=15 /
&MESH ID='mesh17', COLOR='BROWN', IJK=24,28,40, XB=3.8,6.2,1.,3.8,4.,8., MPI_PROCESS=16 /
&MESH ID='mesh18', COLOR='CADET BLUE', IJK=28,28,40, XB=6.2,9.,1.,3.8,4.,8., MPI_PROCESS=17 /


&MISC VERBOSE=.TRUE./
&RADI RADIATION=.FALSE. /

&TIME T_END=400. /


&REAC FUEL = 'METHANE', SOOT_YIELD = 0.2, CO_YIELD = 0.10, RADIATIVE_FRACTION=0.3 /

&SURF ID='BURNER', HRRPUA=4255.5, TAU_MF=0.01 /
&VENT XB=4.25,5.75,4.25,5.75,0.0,0.0, XYZ=5.0,5.0,0.0, RADIUS=0.75, SPREAD_RATE=0.001875, COLOR='RED', SURF_ID='BURNER' /


&VENT MB='XMIN', SURF_ID='OPEN' /  
&VENT MB='XMAX', SURF_ID='OPEN' /  
&VENT MB='YMIN', SURF_ID='OPEN' /  
&VENT MB='YMAX', SURF_ID='OPEN' / 

 
&SPEC ID='WATER VAPOR' /
&PART ID='my droplets', DIAMETER=1000., SPEC_ID='WATER VAPOR' /
&PROP ID='K-11', QUANTITY='SPRINKLER LINK TEMPERATURE', RTI=50., C_FACTOR=0., ACTIVATION_TEMPERATURE=6800., PART_ID='my droplets', FLOW_RATE=300.0, PARTICLE_VELOCITY=10., SMOKEVIEW_ID='sprinkler_pendent' /
&DEVC ID='Spr-1', XYZ=5.0,1.75,7.97, PROP_ID='K-11' /

&DEVC XYZ=5.0,1.75,7.97, QUANTITY='TEMPERATURE', ID='T-1'/
&DEVC XYZ=5.0,1.75,7.97, QUANTITY='VELOCITY', ID='U-1'/


&CTRL ID='kill', FUNCTION_TYPE='KILL', INPUT_ID='delay' /
&CTRL ID='delay', FUNCTION_TYPE='TIME_DELAY', INPUT_ID='Spr-1', DELAY=5. /




&DUMP DT_HRR=1.0, DT_BNDF=1.0, DT_DEVC=1.0 /

&SLCF PBX=5., QUANTITY='TEMPERATURE' /
&SLCF PBZ=2.95, QUANTITY='TEMPERATURE' /
&SLCF PBX=5., QUANTITY='VELOCITY' /


&MISC RESTART=.FALSE.

&TAIL /


\end{lstlisting}
%%%%%%%%%%%%%%%%%%%%%%%%%%%%%%%%%%%%%%%%%%%%%%%%%%
\section{RTI}
Bei der Untersuchung des RTI variiert dieser in Zeile 31 zwischen 27/50/80/120 und 180..
\begin{lstlisting}[emptylines=0,basicstyle=\tiny]
&HEAD CHID='OffenerRaum', TITLE='RTI_27 bis 300s' /

&MESH ID='mesh1', COLOR='MAROON', IJK=56,56,30, XB=1.,3.8,6.2,9.,0.,1.5, MPI_PROCESS=0 /
&MESH ID='mesh2', COLOR='MELON',IJK=48,56,30, XB=3.8,6.2,6.2,9.,0.,1.5, MPI_PROCESS=1 /
&MESH ID='mesh3', COLOR='MINT', IJK=56,56,30, XB=6.2,9.,6.2,9.,0.,1.5, MPI_PROCESS=2 /
&MESH ID='mesh4', COLOR='OLIVE', IJK=56,48,30, XB=1.,3.8,3.8,6.2,0.,1.5, MPI_PROCESS=3 /
&MESH ID='mesh5', COLOR='ORCHID', IJK=56,48,30, XB=6.2,9.,3.8,6.2,0.,1.5, MPI_PROCESS=4 /
&MESH ID='mesh6', COLOR='SALMON', IJK=56,56,30, XB=1.,3.8,1.,3.8,0.,1.5, MPI_PROCESS=5 /
&MESH ID='mesh7', COLOR='STEEL BLUE', IJK=48,56,30, XB=3.8,6.2,1.,3.8,0.,1.5, MPI_PROCESS=6 /
&MESH ID='mesh8', COLOR='FLESH', IJK=56,56,30, XB=6.2,9.,1.,3.8,0.,1.5, MPI_PROCESS=7 /
&MESH ID='mesh9', COLOR='BLUE', IJK=48,48,60, XB=3.8,6.2,3.8,6.2,0.,3., MPI_PROCESS=8 /
&MESH ID='mesh10', COLOR='CHOCOLATE', IJK=56,56,30, XB=6.2,9.,1.,3.8,1.5,3., MPI_PROCESS=9 /
&MESH ID='mesh11', COLOR='COBALT', IJK=56,56,30, XB=1.,3.8,6.2,9.,1.5,3., MPI_PROCESS=10 /
&MESH ID='mesh12', COLOR='HOT PINK',IJK=48,56,30, XB=3.8,6.2,6.2,9.,1.5,3., MPI_PROCESS=11 /
&MESH ID='mesh13', COLOR='KELLY GREEN', IJK=56,56,30, XB=6.2,9.,6.2,9.,1.5,3., MPI_PROCESS=12 /
&MESH ID='mesh14', COLOR='TEAL', IJK=56,48,30, XB=1.,3.8,3.8,6.2,1.5,3., MPI_PROCESS=13 /
&MESH ID='mesh15', COLOR='YELLOW', IJK=56,48,30, XB=6.2,9.,3.8,6.2,1.5,3., MPI_PROCESS=14 /
&MESH ID='mesh16', COLOR='BROWN', IJK=56,56,30, XB=1.,3.8,1.,3.8,1.5,3., MPI_PROCESS=15 /
&MESH ID='mesh17', COLOR='CADET BLUE', IJK=48,56,30, XB=3.8,6.2,1.,3.8,1.5,3., MPI_PROCESS=16 /

&MISC VERBOSE=.TRUE./
&RADI RADIATION=.FALSE. /

&TIME T_END=300. /

&DUMP DT_RESTART=5. /

&REAC FUEL = 'METHANE', SOOT_YIELD = 0.2, CO_YIELD = 0.10, RADIATIVE_FRACTION=0.3 /

&SURF ID='BURNER', HRRPUA=5385.8, TAU_MF=0.01 /
&VENT XB=4.5,5.5,4.5,5.5,0.0,0.0, XYZ=5.0,5.0,0.0, RADIUS=0.5, SPREAD_RATE=0.001666, COLOR='RED', SURF_ID='BURNER' /


&VENT MB='XMIN', SURF_ID='OPEN' /  
&VENT MB='XMAX', SURF_ID='OPEN' /  
&VENT MB='YMIN', SURF_ID='OPEN' /  
&VENT MB='YMAX', SURF_ID='OPEN' / 
 
&SPEC ID='WATER VAPOR' /
&PART ID='my droplets', DIAMETER=1000., SPEC_ID='WATER VAPOR' /
&PROP ID='K-11', QUANTITY='SPRINKLER LINK TEMPERATURE', RTI=27., C_FACTOR=0.0, ACTIVATION_TEMPERATURE=6800., PART_ID='my droplets', FLOW_RATE=300.0, PARTICLE_VELOCITY=10., SMOKEVIEW_ID='sprinkler_pendent' /
&DEVC ID='Spr-1', XYZ=5.0,1.75,2.97, PROP_ID='K-11' /

&DEVC XYZ=5.0,1.75,2.97, QUANTITY='TEMPERATURE', ID='T-1'/
&DEVC XYZ=5.0,1.75,2.97, QUANTITY='VELOCITY', ID='U-1'/


&CTRL ID='kill', FUNCTION_TYPE='KILL', INPUT_ID='delay' /
&CTRL ID='delay', FUNCTION_TYPE='TIME_DELAY', INPUT_ID='Spr-1', DELAY=5. /


&DUMP DT_HRR=1.0, DT_BNDF=1.0, DT_DEVC=1.0 /

&SLCF PBX=5., QUANTITY='TEMPERATURE' /
&SLCF PBZ=2.95, QUANTITY='TEMPERATURE' /
&SLCF PBX=5., QUANTITY='VELOCITY' /


&MISC RESTART=.FALSE.

&TAIL /


\end{lstlisting}



%%%%%%%%%%%%%%%%%%%%%%%%%%%%%%%%%%%%%%%%%



\section{Mesh Sensitivity Study}
\subsection*{Raumhöhe gleich 3~m}
\subsubsection{Zellenlänge gleich 2,5 cm}
\begin{lstlisting}[emptylines=0,basicstyle=\tiny]
&HEAD CHID='OffenerRaum', TITLE='XX' /

&MESH ID='mesh1', COLOR='MAROON', IJK=112,112,120, XB=1.,3.8,6.2,9.,0.,3., MPI_PROCESS=0 /
&MESH ID='mesh2', COLOR='MELON',IJK=96,112,120, XB=3.8,6.2,6.2,9.,0.,3., MPI_PROCESS=1 /
&MESH ID='mesh3', COLOR='MINT', IJK=112,112,120, XB=6.2,9.,6.2,9.,0.,3., MPI_PROCESS=2 /
&MESH ID='mesh4', COLOR='OLIVE', IJK=112,96,120, XB=1.,3.8,3.8,6.2,0.,3., MPI_PROCESS=3 /
&MESH ID='mesh5', COLOR='ORCHID', IJK=112,96,120, XB=6.2,9.,3.8,6.2,0.,3., MPI_PROCESS=4 /
&MESH ID='mesh6', COLOR='SALMON', IJK=112,112,120, XB=1.,3.8,1.,3.8,0.,3., MPI_PROCESS=5 /
&MESH ID='mesh7', COLOR='STEEL BLUE', IJK=96,112,120, XB=3.8,6.2,1.,3.8,0.,3., MPI_PROCESS=6 /
&MESH ID='mesh8', COLOR='FLESH', IJK=112,112,120, XB=6.2,9.,1.,3.8,0.,3., MPI_PROCESS=7 /
&MESH ID='mesh9', COLOR='CYAN', IJK=96,96,60, XB=3.8,6.2,3.8,6.2,0.,1.5, MPI_PROCESS=8 /
&MESH ID='mesh10', COLOR='BLUE', IJK=96,96,60, XB=3.8,6.2,3.8,6.2,1.5,3., MPI_PROCESS=9 /


&MISC VERBOSE=.TRUE./
&RADI RADIATION=.TRUE. /

&TIME T_END=300. /

&DUMP DT_RESTART=5. /

&OBST XB=4.5,5.5,4.5,5.5,0.0,0.05, ID='box' /
&REAC FUEL = 'METHANE', SOOT_YIELD = 0.2, CO_YIELD = 0.10, RADIATIVE_FRACTION=0.2 /

&SURF ID='BURNER', HRRPUA=5500., TAU_MF=0.01 /
&VENT XB=4.5,5.5,4.5,5.5,0.05,0.05, XYZ=5.0,5.0,0.05, RADIUS=0.5, SPREAD_RATE=0.001666, COLOR='RED', SURF_ID='BURNER' /

&SURF ID='Floor', NET_HEAT_FLUX=0. /

&VENT MB='XMIN', SURF_ID='OPEN' /  
&VENT MB='XMAX', SURF_ID='OPEN' /  
&VENT MB='YMIN', SURF_ID='OPEN' /  
&VENT MB='YMAX', SURF_ID='OPEN' / 
&VENT MB='ZMIN', SURF_ID='Floor' /
 
&SPEC ID='WATER VAPOR' /
&PART ID='my droplets', DIAMETER=1000., SPEC_ID='WATER VAPOR' /
&PROP ID='K-11', QUANTITY='SPRINKLER LINK TEMPERATURE', RTI=50., C_FACTOR=0.7, ACTIVATION_TEMPERATURE=68., PART_ID='my droplets', FLOW_RATE=100.0, PARTICLE_VELOCITY=10., SMOKEVIEW_ID='sprinkler_pendent' /
&DEVC ID='Spr-1', XYZ=5.0,1.75,2.97, PROP_ID='K-11' /

&DEVC XYZ=5.0,1.75,2.97, QUANTITY='TEMPERATURE', ID='T-1'/
&DEVC XYZ=5.0,1.75,2.97, QUANTITY='VELOCITY', ID='U-1'/


&CTRL ID='kill', FUNCTION_TYPE='KILL', INPUT_ID='trigger' /
&CTRL ID='trigger', FUNCTION_TYPE='ALL', INPUT_ID='sprinklerActivated', 'delay' /
&CTRL ID='delay', FUNCTION_TYPE='TIME_DELAY', INPUT_ID='sprinklerActivated', DELAY=5. /
&CTRL ID='sprinklerActivated', FUNCTION_TYPE='AT_LEAST', N=1, INPUT_ID='Spr-1' /
&CTRL ID='Spr-1 Activated', FUNCTION_TYPE='AT_LEAST', N=1, INPUT_ID='Spr-1' /

&BNDF QUANTITY='NET HEAT FLUX', PART_ID='box'/

&DUMP DT_HRR=1.0, DT_BNDF=1.0, DT_DEVC=1.0, DT_PL3D=1.0 /
&DUMP PLOT3D_QUANTITY(1:5)='TEMPERATURE', 'U-VELOCITY', 'V-VELOCITY', 'W-VELOCITY', 'HRRPUV' /

&SLCF PBX=5., QUANTITY='TEMPERATURE' /
&SLCF PBZ=2.95, QUANTITY='TEMPERATURE' /
&SLCF PBX=5., QUANTITY='VELOCITY' /


&MISC RESTART=.FALSE.

&TAIL /


\end{lstlisting}

\subsubsection{Zellenlänge gleich 5 cm}
\begin{lstlisting}[emptylines=0,basicstyle=\tiny]
&HEAD CHID='OffenerRaum', TITLE='5,0 300s' /

&MESH ID='mesh1', COLOR='MAROON', IJK=56,56,30, XB=1.,3.8,6.2,9.,0.,1.5, MPI_PROCESS=0 /
&MESH ID='mesh2', COLOR='MELON',IJK=48,56,30, XB=3.8,6.2,6.2,9.,0.,1.5, MPI_PROCESS=1 /
&MESH ID='mesh3', COLOR='MINT', IJK=56,56,30, XB=6.2,9.,6.2,9.,0.,1.5, MPI_PROCESS=2 /
&MESH ID='mesh4', COLOR='OLIVE', IJK=56,48,30, XB=1.,3.8,3.8,6.2,0.,1.5, MPI_PROCESS=3 /
&MESH ID='mesh5', COLOR='ORCHID', IJK=56,48,30, XB=6.2,9.,3.8,6.2,0.,1.5, MPI_PROCESS=4 /
&MESH ID='mesh6', COLOR='SALMON', IJK=56,56,30, XB=1.,3.8,1.,3.8,0.,1.5, MPI_PROCESS=5 /
&MESH ID='mesh7', COLOR='STEEL BLUE', IJK=48,56,30, XB=3.8,6.2,1.,3.8,0.,1.5, MPI_PROCESS=6 /
&MESH ID='mesh8', COLOR='FLESH', IJK=56,56,30, XB=6.2,9.,1.,3.8,0.,1.5, MPI_PROCESS=7 /
&MESH ID='mesh9', COLOR='BLUE', IJK=48,48,60, XB=3.8,6.2,3.8,6.2,0.,3., MPI_PROCESS=8 /
&MESH ID='mesh10', COLOR='CHOCOLATE', IJK=56,56,30, XB=6.2,9.,1.,3.8,1.5,3., MPI_PROCESS=9 /
&MESH ID='mesh11', COLOR='COBALT', IJK=56,56,30, XB=1.,3.8,6.2,9.,1.5,3., MPI_PROCESS=10 /
&MESH ID='mesh12', COLOR='HOT PINK',IJK=48,56,30, XB=3.8,6.2,6.2,9.,1.5,3., MPI_PROCESS=11 /
&MESH ID='mesh13', COLOR='KELLY GREEN', IJK=56,56,30, XB=6.2,9.,6.2,9.,1.5,3., MPI_PROCESS=12 /
&MESH ID='mesh14', COLOR='TEAL', IJK=56,48,30, XB=1.,3.8,3.8,6.2,1.5,3., MPI_PROCESS=13 /
&MESH ID='mesh15', COLOR='YELLOW', IJK=56,48,30, XB=6.2,9.,3.8,6.2,1.5,3., MPI_PROCESS=14 /
&MESH ID='mesh16', COLOR='BROWN', IJK=56,56,30, XB=1.,3.8,1.,3.8,1.5,3., MPI_PROCESS=15 /
&MESH ID='mesh17', COLOR='CADET BLUE', IJK=48,56,30, XB=3.8,6.2,1.,3.8,1.5,3., MPI_PROCESS=16 /


&MISC VERBOSE=.TRUE./
&RADI RADIATION=.FALSE. /

&TIME T_END=300. /


&REAC FUEL = 'METHANE', SOOT_YIELD = 0.2, CO_YIELD = 0.10, RADIATIVE_FRACTION=0.3 /

&SURF ID='BURNER', HRRPUA=5385.8, TAU_MF=0.01 /
&VENT XB=4.5,5.5,4.5,5.5,0.0,0.0, XYZ=5.0,5.0,0.0, RADIUS=0.5, SPREAD_RATE=0.001666, COLOR='RED', SURF_ID='BURNER' /



&VENT MB='XMIN', SURF_ID='OPEN' /  
&VENT MB='XMAX', SURF_ID='OPEN' /  
&VENT MB='YMIN', SURF_ID='OPEN' /  
&VENT MB='YMAX', SURF_ID='OPEN' / 

 
&SPEC ID='WATER VAPOR' /
&PART ID='my droplets', DIAMETER=1000., SPEC_ID='WATER VAPOR' /
&PROP ID='K-11', QUANTITY='SPRINKLER LINK TEMPERATURE', RTI=50., C_FACTOR=0.7, ACTIVATION_TEMPERATURE=6800., PART_ID='my droplets', FLOW_RATE=300.0, PARTICLE_VELOCITY=10., SMOKEVIEW_ID='sprinkler_pendent' /
&DEVC ID='Spr-1', XYZ=5.0,1.75,2.97, PROP_ID='K-11' /

&DEVC XYZ=5.0,1.75,2.97, QUANTITY='TEMPERATURE', ID='T-1'/
&DEVC XYZ=5.0,1.75,2.97, QUANTITY='VELOCITY', ID='U-1'/


&CTRL ID='kill', FUNCTION_TYPE='KILL', INPUT_ID='trigger' /
&CTRL ID='trigger', FUNCTION_TYPE='ALL', INPUT_ID='sprinklerActivated', 'delay' /
&CTRL ID='delay', FUNCTION_TYPE='TIME_DELAY', INPUT_ID='sprinklerActivated', DELAY=5. /
&CTRL ID='sprinklerActivated', FUNCTION_TYPE='AT_LEAST', N=1, INPUT_ID='Spr-1' /
&CTRL ID='Spr-1 Activated', FUNCTION_TYPE='AT_LEAST', N=1, INPUT_ID='Spr-1' /



&DUMP DT_HRR=1.0, DT_BNDF=1.0, DT_DEVC=1.0 /

&SLCF PBX=5., QUANTITY='TEMPERATURE' /
&SLCF PBZ=2.95, QUANTITY='TEMPERATURE' /
&SLCF PBX=5., QUANTITY='VELOCITY' /


&MISC RESTART=.FALSE.

&TAIL /


\end{lstlisting}

\subsubsection{Zellenlänge gleich 10 cm}
\begin{lstlisting}[emptylines=0,basicstyle=\tiny]
 &HEAD CHID='OffenerRaum', TITLE='10_300s' /

&MESH ID='mesh1', COLOR='MAROON', IJK=28,28,15, XB=1.,3.8,6.2,9.,0.,1.5, MPI_PROCESS=0 /
&MESH ID='mesh2', COLOR='MELON',IJK=24,28,15, XB=3.8,6.2,6.2,9.,0.,1.5, MPI_PROCESS=1 /
&MESH ID='mesh3', COLOR='MINT', IJK=28,28,15, XB=6.2,9.,6.2,9.,0.,1.5, MPI_PROCESS=2 /
&MESH ID='mesh4', COLOR='OLIVE', IJK=28,24,15, XB=1.,3.8,3.8,6.2,0.,1.5, MPI_PROCESS=3 /
&MESH ID='mesh5', COLOR='ORCHID', IJK=28,24,15, XB=6.2,9.,3.8,6.2,0.,1.5, MPI_PROCESS=4 /
&MESH ID='mesh6', COLOR='SALMON', IJK=28,28,15, XB=1.,3.8,1.,3.8,0.,1.5, MPI_PROCESS=5 /
&MESH ID='mesh7', COLOR='STEEL BLUE', IJK=24,28,15, XB=3.8,6.2,1.,3.8,0.,1.5, MPI_PROCESS=6 /
&MESH ID='mesh8', COLOR='FLESH', IJK=28,28,15, XB=6.2,9.,1.,3.8,0.,1.5, MPI_PROCESS=7 /
&MESH ID='mesh9', COLOR='BLUE', IJK=24,24,30, XB=3.8,6.2,3.8,6.2,0.,3., MPI_PROCESS=8 /
&MESH ID='mesh10', COLOR='CHOCOLATE', IJK=28,28,15, XB=6.2,9.,1.,3.8,1.5,3., MPI_PROCESS=9 /
&MESH ID='mesh11', COLOR='COBALT', IJK=28,28,15, XB=1.,3.8,6.2,9.,1.5,3., MPI_PROCESS=10 /
&MESH ID='mesh12', COLOR='HOT PINK',IJK=24,28,15, XB=3.8,6.2,6.2,9.,1.5,3., MPI_PROCESS=11 /
&MESH ID='mesh13', COLOR='KELLY GREEN', IJK=28,28,15, XB=6.2,9.,6.2,9.,1.5,3., MPI_PROCESS=12 /
&MESH ID='mesh14', COLOR='TEAL', IJK=28,24,15, XB=1.,3.8,3.8,6.2,1.5,3., MPI_PROCESS=13 /
&MESH ID='mesh15', COLOR='YELLOW', IJK=28,24,15, XB=6.2,9.,3.8,6.2,1.5,3., MPI_PROCESS=14 /
&MESH ID='mesh16', COLOR='BROWN', IJK=28,28,15, XB=1.,3.8,1.,3.8,1.5,3., MPI_PROCESS=15 /
&MESH ID='mesh17', COLOR='CADET BLUE', IJK=24,28,15, XB=3.8,6.2,1.,3.8,1.5,3., MPI_PROCESS=16 /


&MISC VERBOSE=.TRUE./
&RADI RADIATION=.FALSE. /

&TIME T_END=300. /

&REAC FUEL = 'METHANE', SOOT_YIELD = 0.2, CO_YIELD = 0.10, RADIATIVE_FRACTION=0.3 /

&SURF ID='BURNER', HRRPUA=5385.8, TAU_MF=0.01 /
&VENT XB=4.5,5.5,4.5,5.5,0.0,0.0, XYZ=5.0,5.0,0.0, RADIUS=0.5, SPREAD_RATE=0.001666, COLOR='RED', SURF_ID='BURNER' /



&VENT MB='XMIN', SURF_ID='OPEN' /  
&VENT MB='XMAX', SURF_ID='OPEN' /  
&VENT MB='YMIN', SURF_ID='OPEN' /  
&VENT MB='YMAX', SURF_ID='OPEN' / 

 
&SPEC ID='WATER VAPOR' /
&PART ID='my droplets', DIAMETER=1000., SPEC_ID='WATER VAPOR' /
&PROP ID='K-11', QUANTITY='SPRINKLER LINK TEMPERATURE', RTI=50., C_FACTOR=0.7, ACTIVATION_TEMPERATURE=6800., PART_ID='my droplets', FLOW_RATE=300.0, PARTICLE_VELOCITY=10., SMOKEVIEW_ID='sprinkler_pendent' /
&DEVC ID='Spr-1', XYZ=5.0,1.75,2.97, PROP_ID='K-11' /

&DEVC XYZ=5.0,1.75,2.97, QUANTITY='TEMPERATURE', ID='T-1'/
&DEVC XYZ=5.0,1.75,2.97, QUANTITY='VELOCITY', ID='U-1'/


&CTRL ID='kill', FUNCTION_TYPE='KILL', INPUT_ID='trigger' /
&CTRL ID='trigger', FUNCTION_TYPE='ALL', INPUT_ID='sprinklerActivated', 'delay' /
&CTRL ID='delay', FUNCTION_TYPE='TIME_DELAY', INPUT_ID='sprinklerActivated', DELAY=5. /
&CTRL ID='sprinklerActivated', FUNCTION_TYPE='AT_LEAST', N=1, INPUT_ID='Spr-1' /
&CTRL ID='Spr-1 Activated', FUNCTION_TYPE='AT_LEAST', N=1, INPUT_ID='Spr-1' /



&DUMP DT_HRR=1.0, DT_BNDF=1.0, DT_DEVC=1.0 /
&DUMP PLOT3D_QUANTITY(1:5)='TEMPERATURE', 'U-VELOCITY', 'V-VELOCITY', 'W-VELOCITY', 'HRRPUV' /

&SLCF PBX=5., QUANTITY='TEMPERATURE' /
&SLCF PBZ=2.95, QUANTITY='TEMPERATURE' /
&SLCF PBX=5., QUANTITY='VELOCITY' /


&MISC RESTART=.FALSE.

&TAIL /


\end{lstlisting}

\subsubsection{Zellenlänge gleich 20 cm}
\begin{lstlisting}[emptylines=0,basicstyle=\tiny]
&HEAD CHID='OffenerRaum', TITLE='XX20 300s richtig' /

&MESH ID='mesh1', COLOR='MAROON', IJK=14,14,15, XB=1.,3.8,6.2,9.,0.,3., MPI_PROCESS=0 /
&MESH ID='mesh2', COLOR='MELON',IJK=12,14,15, XB=3.8,6.2,6.2,9.,0.,3., MPI_PROCESS=1 /
&MESH ID='mesh3', COLOR='MINT', IJK=14,14,15, XB=6.2,9.,6.2,9.,0.,3., MPI_PROCESS=2 /
&MESH ID='mesh4', COLOR='OLIVE', IJK=14,12,15, XB=1.,3.8,3.8,6.2,0.,3., MPI_PROCESS=3 /
&MESH ID='mesh5', COLOR='ORCHID', IJK=14,12,15, XB=6.2,9.,3.8,6.2,0.,3., MPI_PROCESS=4 /
&MESH ID='mesh6', COLOR='SALMON', IJK=14,14,15, XB=1.,3.8,1.,3.8,0.,3., MPI_PROCESS=5 /
&MESH ID='mesh7', COLOR='STEEL BLUE', IJK=12,14,15, XB=3.8,6.2,1.,3.8,0.,3., MPI_PROCESS=6 /
&MESH ID='mesh8', COLOR='FLESH', IJK=14,14,15, XB=6.2,9.,1.,3.8,0.,3., MPI_PROCESS=7 /
&MESH ID='mesh9', COLOR='CYAN', IJK=12,12,7, XB=3.8,6.2,3.8,6.2,0.,1.4, MPI_PROCESS=8 /
&MESH ID='mesh10', COLOR='BLUE', IJK=12,12,8, XB=3.8,6.2,3.8,6.2,1.4,3., MPI_PROCESS=9 /


&MISC VERBOSE=.TRUE./
&RADI RADIATION=.FALSE. /

&TIME T_END=300. /

&DUMP DT_RESTART=5. /

&REAC FUEL = 'METHANE', SOOT_YIELD = 0.2, CO_YIELD = 0.10, RADIATIVE_FRACTION=0.3 /

&SURF ID='BURNER', HRRPUA=5500., TAU_MF=0.01 /
&VENT XB=4.5,5.5,4.5,5.5,0.0,0.0, XYZ=5.0,5.0,0.0, RADIUS=0.5, SPREAD_RATE=0.001666, COLOR='RED', SURF_ID='BURNER' /


&VENT MB='XMIN', SURF_ID='OPEN' /  
&VENT MB='XMAX', SURF_ID='OPEN' /  
&VENT MB='YMIN', SURF_ID='OPEN' /  
&VENT MB='YMAX', SURF_ID='OPEN' / 

 
&SPEC ID='WATER VAPOR' /
&PART ID='my droplets', DIAMETER=1000., SPEC_ID='WATER VAPOR' /
&PROP ID='K-11', QUANTITY='SPRINKLER LINK TEMPERATURE', RTI=50., C_FACTOR=0.7, ACTIVATION_TEMPERATURE=6800., PART_ID='my droplets', FLOW_RATE=300.0, PARTICLE_VELOCITY=10., SMOKEVIEW_ID='sprinkler_pendent' /
&DEVC ID='Spr-1', XYZ=5.0,1.75,2.97, PROP_ID='K-11' /

&DEVC XYZ=5.0,1.75,2.97, QUANTITY='TEMPERATURE', ID='T-1'/
&DEVC XYZ=5.0,1.75,2.97, QUANTITY='VELOCITY', ID='U-1'/


&CTRL ID='kill', FUNCTION_TYPE='KILL', INPUT_ID='trigger' /
&CTRL ID='trigger', FUNCTION_TYPE='ALL', INPUT_ID='sprinklerActivated', 'delay' /
&CTRL ID='delay', FUNCTION_TYPE='TIME_DELAY', INPUT_ID='sprinklerActivated', DELAY=5. /
&CTRL ID='sprinklerActivated', FUNCTION_TYPE='AT_LEAST', N=1, INPUT_ID='Spr-1' /
&CTRL ID='Spr-1 Activated', FUNCTION_TYPE='AT_LEAST', N=1, INPUT_ID='Spr-1' /



&DUMP DT_HRR=1.0, DT_BNDF=1.0, DT_DEVC=1.0 /


&SLCF PBX=5., QUANTITY='TEMPERATURE' /
&SLCF PBZ=2.95, QUANTITY='TEMPERATURE' /
&SLCF PBX=5., QUANTITY='VELOCITY' /


&MISC RESTART=.FALSE.

&TAIL /


\end{lstlisting}

\subsection*{Raumhöhe gleich 6~m}
\subsubsection{Zellenlänge gleich 5 cm}
\begin{lstlisting}[emptylines=0,basicstyle=\tiny]
&HEAD CHID='OffenerRaum', TITLE='XX5,0_300s' /

&MESH ID='mesh1', COLOR='MAROON', IJK=56,56,60, XB=1.,3.8,6.2,9.,0.,3., MPI_PROCESS=0 /
&MESH ID='mesh2', COLOR='MELON',IJK=48,56,60, XB=3.8,6.2,6.2,9.,0.,3., MPI_PROCESS=1 /
&MESH ID='mesh3', COLOR='MINT', IJK=56,56,60, XB=6.2,9.,6.2,9.,0.,3., MPI_PROCESS=2 /
&MESH ID='mesh4', COLOR='OLIVE', IJK=56,48,60, XB=1.,3.8,3.8,6.2,0.,3., MPI_PROCESS=3 /
&MESH ID='mesh5', COLOR='ORCHID', IJK=56,48,60, XB=6.2,9.,3.8,6.2,0.,3., MPI_PROCESS=4 /
&MESH ID='mesh6', COLOR='SALMON', IJK=56,56,60, XB=1.,3.8,1.,3.8,0.,3., MPI_PROCESS=5 /
&MESH ID='mesh7', COLOR='STEEL BLUE', IJK=48,56,60, XB=3.8,6.2,1.,3.8,0.,3., MPI_PROCESS=6 /
&MESH ID='mesh8', COLOR='FLESH', IJK=56,56,60, XB=6.2,9.,1.,3.8,0.,3., MPI_PROCESS=7 /
&MESH ID='mesh9', COLOR='CYAN', IJK=48,48,30, XB=3.8,6.2,3.8,6.2,0.,1.5, MPI_PROCESS=8 /
&MESH ID='mesh10', COLOR='BLUE', IJK=48,48,30, XB=3.8,6.2,3.8,6.2,1.5,3., MPI_PROCESS=9 /
&MESH ID='mesh11', COLOR='MAROON', IJK=56,56,60, XB=1.,3.8,6.2,9.,3.,6., MPI_PROCESS=10 /
&MESH ID='mesh12', COLOR='MELON',IJK=48,56,60, XB=3.8,6.2,6.2,9.,3.,6., MPI_PROCESS=11 /
&MESH ID='mesh13', COLOR='MINT', IJK=56,56,60, XB=6.2,9.,6.2,9.,3.,6., MPI_PROCESS=12 /
&MESH ID='mesh14', COLOR='OLIVE', IJK=56,48,60, XB=1.,3.8,3.8,6.2,3.,6., MPI_PROCESS=13 /
&MESH ID='mesh15', COLOR='ORCHID', IJK=56,48,60, XB=6.2,9.,3.8,6.2,3.,6., MPI_PROCESS=14 /
&MESH ID='mesh16', COLOR='SALMON', IJK=56,56,60, XB=1.,3.8,1.,3.8,3.,6., MPI_PROCESS=15 /
&MESH ID='mesh17', COLOR='STEEL BLUE', IJK=48,56,60, XB=3.8,6.2,1.,3.8,3.,6., MPI_PROCESS=16 /
&MESH ID='mesh18', COLOR='FLESH', IJK=56,56,60, XB=6.2,9.,1.,3.8,3.,6., MPI_PROCESS=17 /
&MESH ID='mesh19', COLOR='CYAN', IJK=48,48,30, XB=3.8,6.2,3.8,6.2,3.,4.5, MPI_PROCESS=18 /
&MESH ID='mesh20', COLOR='BLUE', IJK=48,48,30, XB=3.8,6.2,3.8,6.2,4.5,6., MPI_PROCESS=19 /

&MISC VERBOSE=.TRUE./
&RADI RADIATION=.FALSE. /

&TIME T_END=300. /

&DUMP DT_RESTART=5. /

&REAC FUEL = 'METHANE', SOOT_YIELD = 0.2, CO_YIELD = 0.10, RADIATIVE_FRACTION=0.3 /

&SURF ID='BURNER', HRRPUA=5386., TAU_MF=0.01 /
&VENT XB=4.5,5.5,4.5,5.5,0.0,0.0, XYZ=5.0,5.0,0.0, RADIUS=0.5, SPREAD_RATE=0.001666, COLOR='RED', SURF_ID='BURNER' /


&VENT MB='XMIN', SURF_ID='OPEN' /  
&VENT MB='XMAX', SURF_ID='OPEN' /  
&VENT MB='YMIN', SURF_ID='OPEN' /  
&VENT MB='YMAX', SURF_ID='OPEN' / 

 
&SPEC ID='WATER VAPOR' /
&PART ID='my droplets', DIAMETER=1000., SPEC_ID='WATER VAPOR' /
&PROP ID='K-11', QUANTITY='SPRINKLER LINK TEMPERATURE', RTI=50., C_FACTOR=0.7, ACTIVATION_TEMPERATURE=6800., PART_ID='my droplets', FLOW_RATE=100.0, PARTICLE_VELOCITY=10., SMOKEVIEW_ID='sprinkler_pendent' /
&DEVC ID='Spr-1', XYZ=5.0,1.75,5.97, PROP_ID='K-11' /

&DEVC XYZ=5.0,1.75,5.97, QUANTITY='TEMPERATURE', ID='T-1'/
&DEVC XYZ=5.0,1.75,5.97, QUANTITY='VELOCITY', ID='U-1'/


&CTRL ID='kill', FUNCTION_TYPE='KILL', INPUT_ID='trigger' /
&CTRL ID='trigger', FUNCTION_TYPE='ALL', INPUT_ID='sprinklerActivated', 'delay' /
&CTRL ID='delay', FUNCTION_TYPE='TIME_DELAY', INPUT_ID='sprinklerActivated', DELAY=5. /
&CTRL ID='sprinklerActivated', FUNCTION_TYPE='AT_LEAST', N=1, INPUT_ID='Spr-1' /
&CTRL ID='Spr-1 Activated', FUNCTION_TYPE='AT_LEAST', N=1, INPUT_ID='Spr-1' /


&DUMP DT_HRR=1.0, DT_BNDF=1.0, DT_DEVC=1.0 /

&SLCF PBX=5., QUANTITY='TEMPERATURE' /
&SLCF PBZ=2.95, QUANTITY='TEMPERATURE' /
&SLCF PBX=5., QUANTITY='VELOCITY' /


&MISC RESTART=.FALSE.

&TAIL /


\end{lstlisting}

\subsubsection{Zellenlänge gleich 10 cm}
\begin{lstlisting}[emptylines=0,basicstyle=\tiny]
&HEAD CHID='OffenerRaum', TITLE='XX10_300s' /

&MESH ID='mesh1', COLOR='MAROON', IJK=28,28,30, XB=1.,3.8,6.2,9.,0.,3., MPI_PROCESS=0 /
&MESH ID='mesh2', COLOR='MELON',IJK=24,28,30, XB=3.8,6.2,6.2,9.,0.,3., MPI_PROCESS=1 /
&MESH ID='mesh3', COLOR='MINT', IJK=28,28,30, XB=6.2,9.,6.2,9.,0.,3., MPI_PROCESS=2 /
&MESH ID='mesh4', COLOR='OLIVE', IJK=28,24,30, XB=1.,3.8,3.8,6.2,0.,3., MPI_PROCESS=3 /
&MESH ID='mesh5', COLOR='ORCHID', IJK=28,24,30, XB=6.2,9.,3.8,6.2,0.,3., MPI_PROCESS=4 /
&MESH ID='mesh6', COLOR='SALMON', IJK=28,28,30, XB=1.,3.8,1.,3.8,0.,3., MPI_PROCESS=5 /
&MESH ID='mesh7', COLOR='STEEL BLUE', IJK=24,28,30, XB=3.8,6.2,1.,3.8,0.,3., MPI_PROCESS=6 /
&MESH ID='mesh8', COLOR='FLESH', IJK=28,28,30, XB=6.2,9.,1.,3.8,0.,3., MPI_PROCESS=7 /
&MESH ID='mesh9', COLOR='CYAN', IJK=24,24,15, XB=3.8,6.2,3.8,6.2,0.,1.5, MPI_PROCESS=8 /
&MESH ID='mesh10', COLOR='BLUE', IJK=24,24,15, XB=3.8,6.2,3.8,6.2,1.5,3., MPI_PROCESS=9 /
&MESH ID='mesh11', COLOR='MAROON', IJK=28,28,30, XB=1.,3.8,6.2,9.,3.,6., MPI_PROCESS=10 /
&MESH ID='mesh12', COLOR='MELON',IJK=24,28,30, XB=3.8,6.2,6.2,9.,3.,6., MPI_PROCESS=11 /
&MESH ID='mesh13', COLOR='MINT', IJK=28,28,30, XB=6.2,9.,6.2,9.,3.,6., MPI_PROCESS=12 /
&MESH ID='mesh14', COLOR='OLIVE', IJK=28,24,30, XB=1.,3.8,3.8,6.2,3.,6., MPI_PROCESS=13 /
&MESH ID='mesh15', COLOR='ORCHID', IJK=28,24,30, XB=6.2,9.,3.8,6.2,3.,6., MPI_PROCESS=14 /
&MESH ID='mesh16', COLOR='SALMON', IJK=28,28,30, XB=1.,3.8,1.,3.8,3.,6., MPI_PROCESS=15 /
&MESH ID='mesh17', COLOR='STEEL BLUE', IJK=24,28,30, XB=3.8,6.2,1.,3.8,3.,6., MPI_PROCESS=16 /
&MESH ID='mesh18', COLOR='FLESH', IJK=28,28,30, XB=6.2,9.,1.,3.8,3.,6., MPI_PROCESS=17 /
&MESH ID='mesh19', COLOR='CYAN', IJK=24,24,15, XB=3.8,6.2,3.8,6.2,3.,4.5, MPI_PROCESS=18 /
&MESH ID='mesh20', COLOR='BLUE', IJK=24,24,15, XB=3.8,6.2,3.8,6.2,4.5,6., MPI_PROCESS=19 /

&MISC VERBOSE=.TRUE./
&RADI RADIATION=.FALSE. /

&TIME T_END=300. /

&DUMP DT_RESTART=5. /


&REAC FUEL = 'METHANE', SOOT_YIELD = 0.2, CO_YIELD = 0.10, RADIATIVE_FRACTION=0.3 /

&SURF ID='BURNER', HRRPUA=5386., TAU_MF=0.01 /
&VENT XB=4.5,5.5,4.5,5.5,0.0,0.0, XYZ=5.0,5.0,0.0, RADIUS=0.5, SPREAD_RATE=0.001666, COLOR='RED', SURF_ID='BURNER' /



&VENT MB='XMIN', SURF_ID='OPEN' /  
&VENT MB='XMAX', SURF_ID='OPEN' /  
&VENT MB='YMIN', SURF_ID='OPEN' /  
&VENT MB='YMAX', SURF_ID='OPEN' / 

 
&SPEC ID='WATER VAPOR' /
&PART ID='my droplets', DIAMETER=1000., SPEC_ID='WATER VAPOR' /
&PROP ID='K-11', QUANTITY='SPRINKLER LINK TEMPERATURE', RTI=50., C_FACTOR=0.7, ACTIVATION_TEMPERATURE=6800., PART_ID='my droplets', FLOW_RATE=100.0, PARTICLE_VELOCITY=10., SMOKEVIEW_ID='sprinkler_pendent' /
&DEVC ID='Spr-1', XYZ=5.0,1.75,5.97, PROP_ID='K-11' /

&DEVC XYZ=5.0,1.75,5.97, QUANTITY='TEMPERATURE', ID='T-1'/
&DEVC XYZ=5.0,1.75,5.97, QUANTITY='VELOCITY', ID='U-1'/


&CTRL ID='kill', FUNCTION_TYPE='KILL', INPUT_ID='trigger' /
&CTRL ID='trigger', FUNCTION_TYPE='ALL', INPUT_ID='sprinklerActivated', 'delay' /
&CTRL ID='delay', FUNCTION_TYPE='TIME_DELAY', INPUT_ID='sprinklerActivated', DELAY=5. /
&CTRL ID='sprinklerActivated', FUNCTION_TYPE='AT_LEAST', N=1, INPUT_ID='Spr-1' /
&CTRL ID='Spr-1 Activated', FUNCTION_TYPE='AT_LEAST', N=1, INPUT_ID='Spr-1' /



&DUMP DT_HRR=1.0, DT_BNDF=1.0, DT_DEVC=1.0 /

&SLCF PBX=5., QUANTITY='TEMPERATURE' /
&SLCF PBZ=2.95, QUANTITY='TEMPERATURE' /
&SLCF PBX=5., QUANTITY='VELOCITY' /


&MISC RESTART=.FALSE.

&TAIL /


\end{lstlisting}

\subsubsection{Zellenlänge gleich 20 cm}
\begin{lstlisting}[emptylines=0,basicstyle=\tiny]
&HEAD CHID='OffenerRaum', TITLE='20_300s' /

&MESH ID='mesh1', COLOR='MAROON', IJK=14,14,15, XB=1.,3.8,6.2,9.,0.,3., MPI_PROCESS=0 /
&MESH ID='mesh2', COLOR='MELON', IJK=12,14,15, XB=3.8,6.2,6.2,9.,0.,3., MPI_PROCESS=1 /
&MESH ID='mesh3', COLOR='MINT', IJK=14,14,15, XB=6.2,9.,6.2,9.,0.,3., MPI_PROCESS=2 /
&MESH ID='mesh4', COLOR='OLIVE', IJK=14,12,15, XB=1.,3.8,3.8,6.2,0.,3., MPI_PROCESS=3 /
&MESH ID='mesh5', COLOR='ORCHID', IJK=14,12,15, XB=6.2,9.,3.8,6.2,0.,3., MPI_PROCESS=4 /
&MESH ID='mesh6', COLOR='SALMON', IJK=14,14,15, XB=1.,3.8,1.,3.8,0.,3., MPI_PROCESS=5 /
&MESH ID='mesh7', COLOR='STEEL BLUE', IJK=12,14,15, XB=3.8,6.2,1.,3.8,0.,3., MPI_PROCESS=6 /
&MESH ID='mesh8', COLOR='FLESH', IJK=14,14,15, XB=6.2,9.,1.,3.8,0.,3., MPI_PROCESS=7 /
&MESH ID='mesh9', COLOR='CYAN', IJK=12,12,30, XB=3.8,6.2,3.8,6.2,0.,6., MPI_PROCESS=8 /
&MESH ID='mesh10', COLOR='CADET BLUE', IJK=14,14,15, XB=6.2,9.,1.,3.8,3.,6., MPI_PROCESS=9 /
&MESH ID='mesh11', COLOR='CHOCOLATE', IJK=14,14,15, XB=1.,3.8,6.2,9.,3.,6., MPI_PROCESS=10 /
&MESH ID='mesh12', COLOR='COBALT',IJK=12,14,15, XB=3.8,6.2,6.2,9.,3.,6., MPI_PROCESS=11 /
&MESH ID='mesh13', COLOR='HOT PINK', IJK=14,14,15, XB=6.2,9.,6.2,9.,3.,6., MPI_PROCESS=12 /
&MESH ID='mesh14', COLOR='KELLY GREEN', IJK=14,12,15, XB=1.,3.8,3.8,6.2,3.,6., MPI_PROCESS=13 /
&MESH ID='mesh15', COLOR='TEAL', IJK=14,12,15, XB=6.2,9.,3.8,6.2,3.,6., MPI_PROCESS=14 /
&MESH ID='mesh16', COLOR='YELLOW', IJK=14,14,15, XB=1.,3.8,1.,3.8,3.,6., MPI_PROCESS=15 /
&MESH ID='mesh17', COLOR='BROWN', IJK=12,14,15, XB=3.8,6.2,1.,3.8,3.,6., MPI_PROCESS=16 /


&MISC VERBOSE=.TRUE./
&RADI RADIATION=.FALSE. /

&TIME T_END=300. /

&DUMP DT_RESTART=5. /


&REAC FUEL = 'METHANE', SOOT_YIELD = 0.2, CO_YIELD = 0.10, RADIATIVE_FRACTION=0.3 /

&SURF ID='BURNER', HRRPUA=5386., TAU_MF=0.01 /
&VENT XB=4.5,5.5,4.5,5.5,0.0,0.0, XYZ=5.0,5.0,0.0, RADIUS=0.5, SPREAD_RATE=0.001666, COLOR='RED', SURF_ID='BURNER' /



&VENT MB='XMIN', SURF_ID='OPEN' /  
&VENT MB='XMAX', SURF_ID='OPEN' /  
&VENT MB='YMIN', SURF_ID='OPEN' /  
&VENT MB='YMAX', SURF_ID='OPEN' / 

 
&SPEC ID='WATER VAPOR' /
&PART ID='my droplets', DIAMETER=1000., SPEC_ID='WATER VAPOR' /
&PROP ID='K-11', QUANTITY='SPRINKLER LINK TEMPERATURE', RTI=50., C_FACTOR=0.7, ACTIVATION_TEMPERATURE=6800., PART_ID='my droplets', FLOW_RATE=100.0, PARTICLE_VELOCITY=10., SMOKEVIEW_ID='sprinkler_pendent' /
&DEVC ID='Spr-1', XYZ=5.0,1.75,5.97, PROP_ID='K-11' /

&DEVC XYZ=5.0,1.75,5.97, QUANTITY='TEMPERATURE', ID='T-1'/
&DEVC XYZ=5.0,1.75,5.97, QUANTITY='VELOCITY', ID='U-1'/


&CTRL ID='kill', FUNCTION_TYPE='KILL', INPUT_ID='trigger' /
&CTRL ID='trigger', FUNCTION_TYPE='ALL', INPUT_ID='sprinklerActivated', 'delay' /
&CTRL ID='delay', FUNCTION_TYPE='TIME_DELAY', INPUT_ID='sprinklerActivated', DELAY=5. /
&CTRL ID='sprinklerActivated', FUNCTION_TYPE='AT_LEAST', N=1, INPUT_ID='Spr-1' /
&CTRL ID='Spr-1 Activated', FUNCTION_TYPE='AT_LEAST', N=1, INPUT_ID='Spr-1' /



&DUMP DT_HRR=1.0, DT_BNDF=1.0, DT_DEVC=1.0 /

&SLCF PBX=5., QUANTITY='TEMPERATURE' /
&SLCF PBZ=2.95, QUANTITY='TEMPERATURE' /
&SLCF PBX=5., QUANTITY='VELOCITY' /


&MISC RESTART=.FALSE.

&TAIL /


\end{lstlisting}

\section{Vergleich mit Grid über Brandherd}
\subsection*{Raumhöhe gleich 3~m}
\begin{lstlisting}[emptylines=0, basicstyle=\tiny]
&HEAD CHID='OffenerRaum', TITLE='H=3m, a=0,047, C=0,5, Vergleich mit einem Grid über Feuer' /

&MESH ID='mesh1', COLOR='MAROON', IJK=56,56,30, XB=1.,3.8,6.2,9.,0.,1.5, MPI_PROCESS=0 /
&MESH ID='mesh2', COLOR='MELON',IJK=48,56,30, XB=3.8,6.2,6.2,9.,0.,1.5, MPI_PROCESS=1 /
&MESH ID='mesh3', COLOR='MINT', IJK=56,56,30, XB=6.2,9.,6.2,9.,0.,1.5, MPI_PROCESS=2 /
&MESH ID='mesh4', COLOR='OLIVE', IJK=56,48,30, XB=1.,3.8,3.8,6.2,0.,1.5, MPI_PROCESS=3 /
&MESH ID='mesh5', COLOR='ORCHID', IJK=56,48,30, XB=6.2,9.,3.8,6.2,0.,1.5, MPI_PROCESS=4 /
&MESH ID='mesh6', COLOR='SALMON', IJK=56,56,30, XB=1.,3.8,1.,3.8,0.,1.5, MPI_PROCESS=5 /
&MESH ID='mesh7', COLOR='STEEL BLUE', IJK=48,56,30, XB=3.8,6.2,1.,3.8,0.,1.5, MPI_PROCESS=6 /
&MESH ID='mesh8', COLOR='FLESH', IJK=56,56,30, XB=6.2,9.,1.,3.8,0.,1.5, MPI_PROCESS=7 /
&MESH ID='mesh9', COLOR='BLUE', IJK=48,48,60, XB=3.8,6.2,3.8,6.2,0.,3., MPI_PROCESS=8 /
&MESH ID='mesh10', COLOR='CHOCOLATE', IJK=56,56,30, XB=6.2,9.,1.,3.8,1.5,3., MPI_PROCESS=9 /
&MESH ID='mesh11', COLOR='COBALT', IJK=56,56,30, XB=1.,3.8,6.2,9.,1.5,3., MPI_PROCESS=10 /
&MESH ID='mesh12', COLOR='HOT PINK',IJK=48,56,30, XB=3.8,6.2,6.2,9.,1.5,3., MPI_PROCESS=11 /
&MESH ID='mesh13', COLOR='KELLY GREEN', IJK=56,56,30, XB=6.2,9.,6.2,9.,1.5,3., MPI_PROCESS=12 /
&MESH ID='mesh14', COLOR='TEAL', IJK=56,48,30, XB=1.,3.8,3.8,6.2,1.5,3., MPI_PROCESS=13 /
&MESH ID='mesh15', COLOR='YELLOW', IJK=56,48,30, XB=6.2,9.,3.8,6.2,1.5,3., MPI_PROCESS=14 /
&MESH ID='mesh16', COLOR='BROWN', IJK=56,56,30, XB=1.,3.8,1.,3.8,1.5,3., MPI_PROCESS=15 /
&MESH ID='mesh17', COLOR='CADET BLUE', IJK=48,56,30, XB=3.8,6.2,1.,3.8,1.5,3., MPI_PROCESS=16 /



&MISC VERBOSE=.TRUE./
&RADI RADIATION=.FALSE. /

&TIME T_END=300. /

&DUMP DT_RESTART=5. /

&REAC FUEL = 'METHANE', SOOT_YIELD = 0.2, CO_YIELD = 0.10, RADIATIVE_FRACTION=0.3 /

&SURF ID='BURNER', HRRPUA=5500., TAU_MF=0.01 /
&VENT XB=4.5,5.5,4.5,5.5,0.0,0.0, XYZ=5.0,5.0,0.0, RADIUS=0.5, SPREAD_RATE=0.001666, COLOR='RED', SURF_ID='BURNER' /


&VENT MB='XMIN', SURF_ID='OPEN' /  
&VENT MB='XMAX', SURF_ID='OPEN' /  
&VENT MB='YMIN', SURF_ID='OPEN' /  
&VENT MB='YMAX', SURF_ID='OPEN' / 
 
&SPEC ID='WATER VAPOR' /
&PART ID='my droplets', DIAMETER=1000., SPEC_ID='WATER VAPOR' /
&PROP ID='K-11', QUANTITY='SPRINKLER LINK TEMPERATURE', RTI=50., C_FACTOR=0.5, ACTIVATION_TEMPERATURE=68., PART_ID='my droplets', FLOW_RATE=300.0, PARTICLE_VELOCITY=10., SMOKEVIEW_ID='sprinkler_pendent' /
&DEVC ID='Spr-1', XYZ=5.0,1.75,2.97, PROP_ID='K-11' /

&DEVC XYZ=5.0,1.75,2.97, QUANTITY='TEMPERATURE', ID='T-1'/
&DEVC XYZ=5.0,1.75,2.97, QUANTITY='VELOCITY', ID='U-1'/


&CTRL ID='kill', FUNCTION_TYPE='KILL', INPUT_ID='trigger' /
&CTRL ID='trigger', FUNCTION_TYPE='ALL', INPUT_ID='sprinklerActivated', 'delay' /
&CTRL ID='delay', FUNCTION_TYPE='TIME_DELAY', INPUT_ID='sprinklerActivated', DELAY=5. /
&CTRL ID='sprinklerActivated', FUNCTION_TYPE='AT_LEAST', N=1, INPUT_ID='Spr-1' /
&CTRL ID='Spr-1 Activated', FUNCTION_TYPE='AT_LEAST', N=1, INPUT_ID='Spr-1' /


&DUMP DT_HRR=1.0, DT_BNDF=1.0, DT_DEVC=1.0 /

&SLCF PBX=5., QUANTITY='TEMPERATURE' /
&SLCF PBZ=2.95, QUANTITY='TEMPERATURE' /
&SLCF PBX=5., QUANTITY='VELOCITY' /



&TAIL /


\end{lstlisting}

\subsection*{Raumhöhe gleich 6~m}
\begin{lstlisting}[emptylines=0, basicstyle=\tiny]
&HEAD CHID='OffenerRaum', TITLE='a=0,047, C=0, H=6, Vergleich Grid über Feuer' /

&MESH ID='mesh1', COLOR='MAROON', IJK=28,28,30, XB=1.,3.8,6.2,9.,0.,3., MPI_PROCESS=0 /
&MESH ID='mesh2', COLOR='MELON',IJK=24,28,30, XB=3.8,6.2,6.2,9.,0.,3., MPI_PROCESS=1 /
&MESH ID='mesh3', COLOR='MINT', IJK=28,28,30, XB=6.2,9.,6.2,9.,0.,3., MPI_PROCESS=2 /
&MESH ID='mesh4', COLOR='OLIVE', IJK=28,24,30, XB=1.,3.8,3.8,6.2,0.,3., MPI_PROCESS=3 /
&MESH ID='mesh5', COLOR='ORCHID', IJK=28,24,30, XB=6.2,9.,3.8,6.2,0.,3., MPI_PROCESS=4 /
&MESH ID='mesh6', COLOR='SALMON', IJK=28,28,30, XB=1.,3.8,1.,3.8,0.,3., MPI_PROCESS=5 /
&MESH ID='mesh7', COLOR='STEEL BLUE', IJK=24,28,30, XB=3.8,6.2,1.,3.8,0.,3., MPI_PROCESS=6 /
&MESH ID='mesh8', COLOR='FLESH', IJK=28,28,30, XB=6.2,9.,1.,3.8,0.,3., MPI_PROCESS=7 /
&MESH ID='mesh9', COLOR='CYAN', IJK=24,24,45, XB=3.8,6.2,3.8,6.2,0.,4.5, MPI_PROCESS=8 /
&MESH ID='mesh10', COLOR='BLUE', IJK=24,24,15, XB=3.8,6.2,3.8,6.2,4.5,6., MPI_PROCESS=9 /
&MESH ID='mesh11', COLOR='CHOCOLATE', IJK=28,28,30, XB=1.,3.8,6.2,9.,3.,6., MPI_PROCESS=10 /
&MESH ID='mesh12', COLOR='COBALT',IJK=24,28,30, XB=3.8,6.2,6.2,9.,3.,6., MPI_PROCESS=11 /
&MESH ID='mesh13', COLOR='HOT PINK', IJK=28,28,30, XB=6.2,9.,6.2,9.,3.,6., MPI_PROCESS=12 /
&MESH ID='mesh14', COLOR='KELLY GREEN', IJK=28,24,30, XB=1.,3.8,3.8,6.2,3.,6., MPI_PROCESS=13 /
&MESH ID='mesh15', COLOR='TEAL', IJK=28,24,30, XB=6.2,9.,3.8,6.2,3.,6., MPI_PROCESS=14 /
&MESH ID='mesh16', COLOR='YELLOW', IJK=28,28,30, XB=1.,3.8,1.,3.8,3.,6., MPI_PROCESS=15 /
&MESH ID='mesh17', COLOR='BROWN', IJK=24,28,30, XB=3.8,6.2,1.,3.8,3.,6., MPI_PROCESS=16 /
&MESH ID='mesh18', COLOR='CADET BLUE', IJK=28,28,30, XB=6.2,9.,1.,3.8,3.,6., MPI_PROCESS=17 /


&MISC VERBOSE=.TRUE./
&RADI RADIATION=.FALSE. /

&TIME T_END=300. /

&DUMP DT_RESTART=5. /


&REAC FUEL = 'METHANE', SOOT_YIELD = 0.2, CO_YIELD = 0.10, RADIATIVE_FRACTION=0.3 /

&SURF ID='BURNER', HRRPUA=5385.8, TAU_MF=0.01 /
&VENT XB=4.5,5.5,4.5,5.5,0.0,0.0, XYZ=5.0,5.0,0.0, RADIUS=0.5, SPREAD_RATE=0.001666, COLOR='RED', SURF_ID='BURNER' /


&VENT MB='XMIN', SURF_ID='OPEN' /  
&VENT MB='XMAX', SURF_ID='OPEN' /  
&VENT MB='YMIN', SURF_ID='OPEN' /  
&VENT MB='YMAX', SURF_ID='OPEN' / 

 
&SPEC ID='WATER VAPOR' /
&PART ID='my droplets', DIAMETER=1000., SPEC_ID='WATER VAPOR' /
&PROP ID='K-11', QUANTITY='SPRINKLER LINK TEMPERATURE', RTI=50., C_FACTOR=0.0, ACTIVATION_TEMPERATURE=68., PART_ID='my droplets', FLOW_RATE=300.0, PARTICLE_VELOCITY=10., SMOKEVIEW_ID='sprinkler_pendent' /
&DEVC ID='Spr-1', XYZ=5.0,1.75,5.97, PROP_ID='K-11' /

&DEVC XYZ=5.0,1.75,5.97, QUANTITY='TEMPERATURE', ID='T-1'/
&DEVC XYZ=5.0,1.75,5.97, QUANTITY='VELOCITY', ID='U-1'/


&CTRL ID='kill', FUNCTION_TYPE='KILL', INPUT_ID='trigger' /
&CTRL ID='trigger', FUNCTION_TYPE='ALL', INPUT_ID='sprinklerActivated', 'delay' /
&CTRL ID='delay', FUNCTION_TYPE='TIME_DELAY', INPUT_ID='sprinklerActivated', DELAY=5. /
&CTRL ID='sprinklerActivated', FUNCTION_TYPE='AT_LEAST', N=1, INPUT_ID='Spr-1' /
&CTRL ID='Spr-1 Activated', FUNCTION_TYPE='AT_LEAST', N=1, INPUT_ID='Spr-1' /



&DUMP DT_HRR=1.0, DT_BNDF=1.0, DT_DEVC=1.0 /

&SLCF PBX=5., QUANTITY='TEMPERATURE' /
&SLCF PBZ=2.95, QUANTITY='TEMPERATURE' /
&SLCF PBX=5., QUANTITY='VELOCITY' /


&MISC RESTART=.FALSE.

&TAIL /


\end{lstlisting} % Technische Ergänzungen
\chapter{Ergänzende Inhalte} % \chapter{Inhalt der CD-ROM/DVD}
\label{app:materials}


Auflistung der ergänzenden Materialien zu dieser Arbeit, die zur digitalen Archivierung an der 
Hochschule eingereicht wurden (als ZIP-Datei).

% Nur als Beispiel, die Struktur sollte man an die eigenen Bedürfnisse anpassen!

\section{PDF-Dateien}
\begin{FileList}{/}
\fitem{thesis.pdf} Finale Master-/Bachelorarbeit (Gesamtdokument)
\end{FileList}

\section{Mediendaten}
\begin{FileList}{/media}
\fitem{*.ai, *.pdf} Adobe Illustrator-Dateien
\fitem{*.jpg, *.png} Rasterbilder
\fitem{*.mp3} Audio-Dateien
\fitem{*.mp4} Video-Dateien
\end{FileList}


\section{Online-Quellen (PDF-Kopien)}
\begin{FileList}{/online-sources}
\fitem{Reliquienschrein-Wikipedia.pdf} {\backtrackerfalse\parencite{WikiReliquienschrein2023}}
\end{FileList}
 % Inhalt der CD-ROM/DVD
\chapter{Chronologische Liste der Änderungen}


\begin{sloppypar}
\begin{description}
%
\item[2002/01/07]
\verb!\newfloat{program}! repariert (auch ohne Chapter). Dank an Werner Bailer!
%
\item[2002/03/06]
Copyright-Notice an internat.\ Standard angepasst. Dank an Karin Kosina!
%
\item[2002/07/28]
"`Studiengang"' $\rightarrow$ "`Diplomstudiengang"'
%
\item[2003/08/24]
Neues Macro: \verb!\Messbox{breite}{hoehe}! -- zur Kontrolle der 
Druckgröße ohne PS-Datei. Erweiterungen für Bakkalaureatsarbeiten
%
\item[2005/04/09]
Diverse Korrekturen: Captions von Tabellen nach oben gesetzt. 
\texttt{caption} auf neue Versionen adaptiert.
\texttt{subfigure} wird nicht mehr verwendet
%
\item[2006/01/20]
Adaptiert zur Verwendung als Praktikumsbericht 
(2.\ Bakk.-Arbeit)
%
\item[2006/03/24]
Fehler in \verb!\erklaerung! beseitigt (Dank an David Schwingenschlögl)
%
\item[2006/04/06]
Verwendung von T1-Fontencoding zur besseren Silbentrennung bei 
Umlauten etc.
%
\item[2006/06/21]
Neu: Bachelorstudiengang / Masterstudiengang.
Literaturverweise auf Bakk-Arbeiten.
\texttt{upquote.sty} eliminiert (Problem mit TS1-Kodierung).
Verwende Komma (statt Punkt) als Trennzeichen in Dezimalzahlen.
%
\item[2006/09/14]
Anmerkungen zum Thema Plagiarismus.
%
\item[2007/07/16]
Ergänzungen für Code-Listings (listings) und Algorithmen 
(\texttt{algorithmicx}).
BiBTeX-Datei aufgeräumt, Verwendung der Literaturformate 
verbessert.
Komma (statt Punkt) als Trennzeichen in Dezimalzahlen wieder 
entfernt.
Verwendung der \texttt{ae}-Fonts eliminiert (\texttt{cm-super} Fonts müssen 
installiert sein, ab MikTeX 2.5). 
Beispiel für Ersetzung in EPS-Dateien mit \texttt{psfrag}.
%
\item[2007/10/04]
Version 5.90: Das Laden der Pakete \verb!inputenc! (Option \texttt{latin}) und 
\verb!graphicx! (Option \texttt{dvips})
aus der Hauptdatei in die \texttt{sty}-Datei übertragen; \texttt{upquote} funktioniert nun.
Paket \texttt{eurosym} ergänzt für Euro-Symbol (Anregung von Andreas 
Doubrava).
Problem mit \texttt{color}-package repariert (gerasterter PDF-Ausdruck).
Hinweise bzgl.\ Literatur ergänzt (\texttt{month}, \texttt{edition}),
BibTeX-Datei gesäubert.
Hinweis zum Einfügen von vertikalem Abstand zwischen Absätzen.
Mathematik aufgeräumt, Verwendung von \texttt{amsmath}, 
Fallunterscheidungen.
Diverse Änderungen bei Tabellen und Programmkode.
Beispiele für BibTeX-Angaben von Spezialquellen: Audio-CDs, 
Videos, Filme. Einbinden von Dateien mit \verb!\include{..}!
Neue Datei: \verb!_SimpleReport.tex! für kurze Reports (Projekte etc.).
%
\item[2007/11/11]
Version 5.91: Hinweise zur Einstellung der Output-Profile in
TexNicCenter, Inverse Search Einstellung in YAP im Anhang.
%
\item[2008/04/01]
Version 6.00beta -- kompletter Umbau!
Auslagerung der Doku\-menten-relevanten Teile in eine eigene 
\emph{class}-Datei (\texttt{hgbthesis.cls}) mit Optionen.
Die neue Style-Datei \texttt{hgb.sty} ist nun unabhängig vom 
Dokumententyp und nicht mehr kompatibel mit älteren Versionen!
Die Liste der Änderungen ist jetzt in der Datei \verb!_ChangeLog.tex!
(DIESE Datei) und diese wird im Anhang eingebunden.
Heading-Style auf Sans Serif geändert (ohne grausliche "`Caps"').
%
\item[2008/05/22]
Neue Vorlage für Technical Reports (Klasse \texttt{hgbreport.cls}).
Spracheinstellung nunmehr mit \texttt{babel}-Paket, Hauptsprache
des Dokuments kann als Option der Klasse angegeben werden.
Sprachumschaltung innerhalb des Dokuments funktioniert nun
richtig. Mit der Sprachoption \texttt{german} wird automatisch die neue deutsche 
Orthographie (\texttt{ngerman}) verwendet.
\texttt{babelbib} wird zur Formatierung des Literaturverzeichnisses
verwendet (neue BibTeX-Style-Optionen!).
Header werden nunmehr mit \texttt{fancyhdr}-Paket erzeugt.
Versionsnummerierung von \texttt{.cls} und \texttt{.sty} Files wird beendet 
(ab jetzt gilt: \emph{Datum} = \emph{Version}). 
%
\item[2008/06/10]
Neues Listing-Environment \texttt{PhpCode}; bei allen Listing-Eviron\-ments ist nun 
\texttt{mathescape=false} (kein Math-Mode nach \verb!$!). 
Bug bei Sprachumschaltung auf \texttt{ngerman} beseitigt.
%
\item[2008/08/15]
Diverse Kleinigkeiten in Literaturangaben überarbeitet (Dank an Norbert Wenzel), Spracheinstellung vereinheitlicht, Umlaute in \texttt{.bib}-Datei ersetzt.
%
\item[2008/10/15] 
Zusätzliche Hinweise zur MikTeX-Installation (Windows) sowie LaTeX unter Mac OS~X und Linux.
Liste der Abkürzungen ergänzt.%
\item[2008/11/15] 
Diverse Schreibfehler korrigiert (Dank an Silvia Fuchshuber). Hinweis auf 
\texttt{sloppypar}-Umgebung.
%
\item[2008/12/09] 
BibTeX-Tools: neuer Hinweis auf JabRef ergänzt, BibEdit entfernt (ist nicht mehr auffindbar).
%
\item[2009/02/09]
\texttt{hgb.sty}: Option "`\texttt{spaces}"' zu \texttt{url}-Package ergänzt (ermöglicht gezielten Zeilenumbruch in URLs). 
Im allgemeinen Setup für \texttt{listings}: \texttt{keepspaces=true};
Obsoletes Environment \texttt{sourcecode} deaktiviert.
Escape-Mode für \texttt{LaTeXCode}-Umgebung geändert.
\verb!_DaBa.tex!: Hinweis auf die Verwendung von \verb!\urldef! für die Angabe von URLs in Captions. \texttt{diplom} (statt \texttt{master}) als Standard-Dokumententyp in \verb!_DaBa.tex! ("`Diplomarbeit"'). Neuer Abschnitt zum Umgang mit ``Quellenangaben in Captions''.
\texttt{literatur.bib}: alle URLs (bisher in \texttt{note}-Einträgen) auf \verb!url={..}! geändert.
%
\item[2009/04/14]
Hinweis zum Einfügen einfacher Anführungszeichen ergänzt.
%
\item[2009/07/18]
Literaturangaben korrigiert und ergänzt.
%
\item[2009/11/27]
Experimentelle Version: Massive Änderungen, Umstieg auf \texttt{pdflatex}.
%
\item[2010/06/15]
Erstes Release der neuen Version mit \texttt{pdflatex}.
\item[2010/06/23]
Konflikt zwischen \texttt{pdfsync}-Package und \texttt{array}-Package (wird relativ häufig benutzt) durch \verb!\RequirePackage[novbox]{pdfsync}! behoben.
Seitenunterkante durch \verb!\flushbottom! fixiert,
variablen Absatzzwischenraum reduziert.
\item[2010/07/27]
Sprache der Erklärungsseite auf "`\texttt{german}"' fixiert (auch wenn die Hauptsprache des Dokuments  Englisch ist). %Datumsproblem - Hinweis von Philipp Winter
\item[2010/12/03]
Anmerkungen und Beispiele zum Zitieren von Gesetzestexten und Videos (Zeitangabe) ergänzt.
Hinweis auf \verb!\nolinkurl{..}! zur Angabe von Dateinamen.
\item[2011/01/29]
Einbau der Creative Commons Lizenz und entsprechender Hinweis in 
Abschnitt \ref{sec:HagenbergEinstellungen}. Neue Makros
\verb!\strictlicense!,
\verb!\cclicense! und
\verb!\license{...}!.
BibTeX-Einträge für Audio-CDs und Filme korrigiert, Beispiel für Online-Video ergänzt.
\item[2011/02/01]
Neues Makro \verb!\betreuerin{..}! zur Angabe einer (weiblichen) Betreuerin. 
%
\item[2011/06/26]
Umstellung der gesamten Literaturverwaltung auf \texttt{biblatex} mit dem Ziel, 
getrennte Abschnitte für verschiedene Kategorien von Einträgen im Quellenverzeichnis
zu ermöglichen. Die Wahl fiel auf \texttt{biblatex} (es gäbe andere Optionen), weil
damit BibTeX weiterhin nur einmal aufgerufen werden muss (und nicht für
mehrere Dateien). Damit verbunden sind allerdings massive Änderungen bei der
Syntax der BibTeX-Felder und es gibt auch mehrere neue Felder.
Aktuell sind 3 Kategorien von Quellen vorgesehen, entsprechende Änderungen in 
\nolinkurl{hgbthesis.cls}. Der klassische BibTeX-Workflow wird aktuell nicht
mehr unterstützt, die Möglichkeit einer künftigen Dok-Option ist aber 
vorgesehen. Das Literatur-Kapitel ist komplett überarbeitet, die .bib-Datei
wurde ausgemistet. Neu ist die Empfehlung zur Aufnahme von Bildquellen
in das Quellenverzeichnis, womit lange URLs in Captions (dort sind keine
Fußnoten möglich) nicht mehr notwendig sind. 
"`Persönliche Kommunikation"' als Literaturquelle entfernt (den Inhalt
von Interviews sollte man direkt im Anhang wiedergeben).
Das verwendete Bildmaterial wurde
erneuert, aktuell werden nur mehr Public Domain Bilder verwendet. 
Das Kapitel "`Hinweise für Word-Benutzer"' wurde endgültig entfernt.
\verb!\flushbottom! wieder auf \verb!\raggedbottom! geändert, um übermäßige 
Abstände zwischen Absätzen zu vermeiden.
%
\item[2012/05/10]
Hinweis auf die in Österreich bislang nicht zulässige Verwendung von "`Masterarbeit"'
entfernt, \texttt{master} ist nunmehr die Default-Dokumentenoption.
Anmerkungen zu lästigen \texttt{biblatex}-Warnungen ergänzt.
Angaben für Windows-Programmpfade auf Win7 angepasst, 
MikTeX 2.9 als Minimalerfordernis.\newline
Überflüssige Makros \verb!\damonat! und \verb!\dajahr! endgültig entfernt, statt
\verb!\abgabemonat! und \verb!\abgabejahr! ist nun das neue Makro
\verb!\abgabedatum{yyyy}{mm}{dd}! vorgesehen (unter Verwendung von internen Zählern).
Zur Formatierung von Datumsangaben wir das \texttt{datetime}-Paket verwendet.
\newline
Neue Fassung der eidesstattlichen Erklärung (inkl.\ englischer Version).\newline
PDF-Suche auf \texttt{synctex} umgestellt (\texttt{pdfsync}-Paket ist veraltet und
wird nun nicht mehr verwendet).
\newline
Die älteren Dateiversionen von \texttt{algorithmicx.sty} und \texttt{alg\-pseudo\-code.sty}
(bisher explizit beigefügt) wurden weggelassen.
\newline
Hinweis auf die \emph{Latin Modern Roman} OTF-Schriften ergänzt.
%
\item[2012/07/21]
Quellenverzeichnis: sprachabhängige Einstellung der Überschriften eingerichtet.
Titel des Quellenverzeichnisses auf "`Quellenverzeichnis"' (DE) \bzw\ "`References"' (EN) 
fixiert. Makro \verb!\MakeBibliography! hat damit keinen erforderlichen Parameter mehr.
%
\item[2012/09/17]
Wegen Änderungen im \texttt{biblatex}-package (Version 1.7, 2011/11/13) die Verwendung von
BibTeX als backend eingestellt (\texttt{backend=bibtex8}).
%
\item[2012/10/13]
Option \texttt{lowtilde} beim URL-package eingestellt (erzeugt \url{~} statt \verb!~!).
%
\item[2012/12/01]
In Abschnitt \ref{sec:FormatierungVonProgrammcode} zusätzliche Code-Umgebungen ergänzt:
\texttt{JsCode},
\texttt{PhpCode},
\texttt{HtmlCode},
\texttt{CssCode},
\texttt{XmlCode}.
%
\item[2012/12/08]
Die Code-Umgebungen in Abschn.\ \ref{sec:FormatierungVonProgrammcode} ergänzt und 
zur Verwendung von optionalen Argumenten erweitert (Hinweise in Abschnitt 
\ref{sec:FormatierungVonProgrammcode} auf die Argumente
\texttt{firstnumber=last} und \texttt{numbers=none}).
Quellenverzeichnis: den Eintragstyp \texttt{@software} für Games empfohlen und im Verzeichnis
der Kategorie \emph{avmedia} zugeordnet (Tab.~\ref{tab:BibKategorien} ergänzt). 
Game-Beispiel (von Manuel Wieser) und zusätzliche Tabelle \ref{tab:QuellenUndEintragstypen}
zur besseren Übersicht eingefügt.
%
\item[2013/05/17]
Wichtigste Änderung ist die vollständige Umstellung auf \textbf{UTF-8} unter Beibehaltung des 
\texttt{pdflatex}-Workflows. 
Damit sind zahlreiche weitere Modifikationen verbunden:
\newline
Alle Dateien (auch \texttt{.cls}, \texttt{.sty} und \texttt{.bib}) wurden auf UTF-8 konvertiert.
Damit sollte es auch keine Probleme mehr mit Umlauten und Sonderzeichen unter MacOS geben.
\newline
Die verwendete Standard-Schriftfamilie ist nun "`Latin Modern"' (\texttt{lmodern}). 
Sie ersetzt die "`CM-Super"' Schriften, mit denen es immer wieder Installationsprobleme gab.
Weiters wird jetzt das \texttt{cmap}-Paket zur besseren Such- und Kopierbarkeit von PDFs verwendet.
\newline
Das \texttt{listings}-Paket wurde durch \texttt{listingsutf8} ersetzt und für Umlaute im Quellcode adaptiert.
Eventuell sind weitere Adaptierungen notwendig.
\newline
\texttt{biber} (min.\ Version 1.5!) wird nun anstatt \texttt{bibtex} (unterstützt keine UTF-8 Dateien) verwendet,
zusammen mit \texttt{biblatex} (Version 2.5).
Die Anweisung \verb!\bibliography! wird (wieder) verwendet, allerdings nun in der Präambel,
um die \texttt{.bib}-Datei im Fileverzeichnis anzuzeigen.
\newline
Das Makro \verb!\C! (für die Menge der komplexen Zahlen \Cpx) musste wegen Problemen in der T1-Kodierung
ersetzt werden und heißt nun \verb!\Cpx!. Die Makros 
\verb!\R!, \verb!\Z!, \verb!\N!, \verb!\Q! und \verb!\Cpx! können nun auch außerhalb des Mathematik-Modus verwendet werden.
\newline
Der DVI-PS-PDF Workflow wird ab dieser Version überhaupt nicht mehr unterstützt. 
Damit ist auch das \texttt{psfrag}-Paket nicht mehr verwendbar. Entspechende Hinweise 
wurden aus dem Text entfernt.
\newline
\texttt{hyperref} wurde auf UTF-8 umgestellt.
Die grässlichen Standard-Rahmen und Farben der automatischen \texttt{hyperref}-Links wurden entfernt \bzw\ durch 
dezentere Farben ersetzt. Dadurch wird auch die Screen-Version der PDFs wieder lesbar.
\newline
Im Quellenverzeichnis wurde versuchsweise die \texttt{backref}-Option aktiviert. 
Damit werden bei allen Einträgen auch die zugehörigen Zitierstellen angegeben
(erscheint durchaus sinnvoll).
\newline
Die bisherigen Korrekturen zur \texttt{biblatex}-Formatierung wurden entfernt, 
alles arbeitet nun mit Standard-Einstellungen. Die ursächlichen Probleme in \texttt{biblatex}
scheinen in der aktuellen Version behoben zu sein.
\newline
Das Output-Profil für TeXnicCenter wurde für den neuen Workflow mit \texttt{biber} adaptiert und liegt nun in
\nolinkurl{_tc_output_profile_sumatra_utf8.tco}.
\newline
Das Windows-Script \verb!_clean.bat! wurde entfernt, da TeXnicCenter nun ein eigenes "`Clean Project"'-Kommando aufweist (in "`Build"').
\newline
Allgemeine Einstellungen zu \emph{headings} und \emph{biblatex} wurden aus der Datei \texttt{hgbthesis.cls} entfernt und in 
\texttt{hgbheadings.sty} \bzw\ \texttt{hgbbib.sty} verlagert. Diese können nun unabhängig verwendet werden (s.\ Beispiel in 
\texttt{\_TermReport.tex}).
\newline
Die Klassen-Datei \texttt{hgbtermreport.cls} wurde eliminiert, das Dokument \texttt{\_TermReport.tex} basiert nunmehr
auf der generischen LaTeX-Klasse \texttt{report}  und verwendet keine eigene \texttt{.cls} Datei mehr.
%
\item[2014/11/05]
Neu: Logo auf der Frontseite bei allen Dokumententypen. Dazu gibt es ein neues Kommando
\verb!\logofile{pic}!, wobei \verb!pic! der Name eine PDF-Datei im
Verzeichnis \verb!images/! ist. Falls \emph{kein} Logo erwünscht ist, 
kann man die Zeile einfach weglassen oder durch \verb!\logofile{}! ersetzen.
\newline
\texttt{hyperref}-Einstellungen: Einfärbung der Links wieder entfernt (\texttt{colorlinks = false}), weil beim Druck
nicht abschaltbar. Stattdessen einheitliche (dezente) Rahmen für alle Linkarten.
Zahlreiche Tippfehler eliminiert (Dank an Daniel Karzel).
\newline
Wegen eines Bugs in \texttt{biblatex 1.9} wurden die expliziten Abteilungen (\verb!\-!) in \texttt{literatur.bib}
vorübergehend entfernt (mit entsprechenden Folgen im Ergebnis). Der Bug soll in \texttt{biblatex 2.0} (derzeit noch
nicht verfügbar) behoben sein.
\newline
Package \texttt{color} auf \texttt{xcolor} geändert. In \texttt{hgb.sty} neues "`Convenience-Makro"' \verb!\etc! ergänzt.
Output-Profil für TeXnicCenter/SumatraPDF (Windows) repariert, forward/inverse Search funktioniert nun
(Datei \verb!_tc_output_profile_sumatra_utf8.tco!).
%
\item[2015/04/28]
Paket \texttt{subdepth} (zur verbesserten Platzierung von Sub- und Superscripts) 
in hgb.sty ergänzt.
%
\item[2015/07/14]
Hinweis und Abhilfe für die (nicht automatische) Silbentrennung in zusammengesetzten Wörtern.
Neu in \texttt{hgbheadings.sty}: \verb!\RequirePackage[raggedright]{titlesec}! verhindert Blocksatz
in Section-Überschriften (sehr unschön bei längeren Überschriften). 
Neu (in Abschn.~\ref{sec:GraphicOverlays}): Beispiel für die Verwendung des \texttt{overpic}-Pakets
zur Annotierung von importierten Grafiken (verwendet zudem das \texttt{pict2e}-Paket).
%
\item[2015/08/03]
Logo-Datei auf \texttt{logo.pdf} umbenannt.
\item[2015/09/17]
Anweisung \verb!\RequirePackage[utf8]{inputenc}! in die Doku\-menten\-dateien (\texttt{\_xxx.tex})
verschoben (auf Anregung von Markus Kohm: "`\ldots für die Verwendung von lualatex oder xelatex 
ist die Anweisung in hgb.sty störend, da bei diesen beiden aufgrund der nativen utf8-Unterstützung 
\texttt{inputenc} keinesfalls verwendet werden darf"').
\item[2015/09/19]
\texttt{hgb.sty} aufgeräumt.
Makros \verb!\@savesymbol! und \verb!\@restoresymbol! aus \texttt{hgb.sty} entfernt
(wurden nicht mehr verwendet; ggfs.\ Paket \texttt{savesym} als Ersatz).
Makro \verb!\optbreaknh! (optional break with no hyphen) auf \verb!\obnh! umbenannt.
Teile von \texttt{hgb.sty} in neue Dateien \texttt{hgbabbrev.sty} (div.\ Abkürzungen)
und \texttt{hgblistings.sty} (Code-Listings) verschoben.
Hintergrundtönung der Code-Listings heller (auf 5\% Grau) eingestellt.
Layout: \verb!\textfraction! auf 0.1 (statt fehlhafterweise 0.01) eingestellt.
\texttt{hgbbib.sty}: \verb!\clearpage! am Beginn des Quellenverzeichnisses entfernt
(für \texttt{article}-Template).
\item[2015/09/19]
Alle \texttt{.cls} und \texttt{.sty} Dateien sind jetzt ANSI-codiert (Header eingefügt), wie
laut CTAN-Richtlinien vorgesehen. Umlautzeichen wurden durch Makros ersetzt.
Nur \texttt{hgblistings.sty} ist weiterhin UTF-8 (wegen notwendiger literaler Umlaute).
\verb!\RequirePackage[utf8]{inputenc}! steht sonst nur mehr am Beginn
der jeweiligen (\texttt{.tex}) Haupttextdatei.
\item[2015/10/29]
Verwendung von "`In:"' im Quellenverzeichnis vor \texttt{article}-Einträgen
(Eigenart von biblatex) durch passendes Makro in \texttt{hgbbib.sty} unterbunden 
(Dank an S.\ Dreiseitl).
\item[2015/11/04]
Hinweise in Abschnitt \ref{sec:Software} auf TeXstudio unter Windows, Mac OS und Linux.
Release-Ausgabe.
\item[2015/12/08]
Source Directories neu strukturiert in \texttt{frontmatter}, \texttt{chapters}, 
\texttt{appendix}.
\item[2016/06/09]
Bibliography-Aliases für die Quellentypen
\texttt{video}, \texttt{movie}, \texttt{audio} und \texttt{software}
eingefügt (in \texttt{hgbbib.sty}) -- unterbindet Warnungen wegen
fehlender biblatex-Driver.
\item[2016/06/11]
Repository portiert auf GitHub (SourceForge eingefroren).  
Overleaf als experimentelle online LaTeX-Umgebung.
Hauptdateien umbenannt (auf \texttt{\_thesis}, \texttt{\_praktikum}, etc.).
\end{description}
\end{sloppypar}




%\section*{To Do} 
%\begin{itemize}
%\item Inkscape
%\item biblatex Bib-Driver für audio, video etc. ergänzen.
%\item Mathematik umbauen, typische Fehler stärker berücksichtigen (ua. Leerzeilen vor/nach Gleichungen).
%\item Literaturempfehlungen zum Schreiben von Diplomarbeiten
%\item Hinweise für Literatursuche (Bibliotheksverbund, CiteSeer,...)
%\end{itemize}





 % Chronologische Liste der Änderungen
\chapter{\latex-Quellcode}
\label{app:Quellcode}

 % Quelltext dieses Dokuments

%%%-----------------------------------------------------------------------------
\backmatter                          % Schlussteil (Quellenverzeichnis und dgl.)
%%%-----------------------------------------------------------------------------

\MakeBibliography % Quellenverzeichnis

%%%-----------------------------------------------------------------------------
% Messbox zur Druckkontrolle
%%%-----------------------------------------------------------------------------

\phantomsection
\chapter*{Messbox zur Druckkontrolle}



\begin{center}
{\Large --- Druckgröße kontrollieren! ---}

\bigskip

\Messbox{100}{50} % Angabe der Breite/Hoehe in mm

\bigskip

{\Large --- Diese Seite nach dem Druck entfernen! ---}

\end{center}



%%%-----------------------------------------------------------------------------
\end{document}
%%%-----------------------------------------------------------------------------